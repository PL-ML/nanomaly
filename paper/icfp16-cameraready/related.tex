\section{Related Work}
\label{sec:related-work}
In this section we connect our work to related efforts on type errors,
testing, and program exploration.

% \paragraph{Type Errors}
% \label{sec:type-error}

\paragraph{Localizing and Repairing Type Errors}
\label{sec:diagnosis-repair}
It is well known that unification-based type inference procedures can
produce poor error messages, and in particular, can misidentify the
\emph{source} of the type error.
%
Thus, many groups have explored techniques to pinpoint the true source
of the error and recommend fixes.

% \subparagraph{Localization}
The traditional Damas-Milner type inference
algorithm~\cite{Damas1982-uw} reports the first program location where a
type mismatch is discovered (subject to the traversal
strategy~\cite{Lee1998-ys}).
%
As a result the error can be reported far away from its
source~\cite{McAdam1998-ub} without enough information to guide the
user.
%
Type-error slicing~\cite{Haack2003-vc,Schilling2011-yf,Rahli2015-tt,Sagonas2013-bf,Gast2004-zd,Neubauer2003-xv}
%treats type inference as a constraint-satisfaction problem and
recognizes this flaw and instead produces a slice of the program
containing \emph{all} program locations that are connected to the type
error.
%
%% \citet{Neubauer2003-xv} present a decidable type system based on
%% discriminative sum types, in which all terms are typeable and type
%% derivations contain all type errors in a program. They then use the
%% typing derivation to slice out the parts of the expression related to
%% each error.
%
Though the program slice must contain the source of the error, it can
suffer from the opposite problem of providing \emph{too much}
information, motivating recent work in ranking the candidate locations.
%
Zhang~\etal~\cite{Zhang2014-lv,Zhang2015-yu} present an algorithm for
identifying the most likely culprit using Bayesian reasoning.
%
\citet{Pavlinovic2014-mr,Pavlinovic2015-kh} translate the %error
localization problem to a MaxSMT optimization problem, using
compiler-provided weights to rank the possible sources.

% \subparagraph{Repairing}
In addition to localizing the error, \citet{Lerner2007-dt} attempt to
suggest a fix by replacing expressions (or removing them entirely) with
alternatives based on the surrounding program context.
%
\citet{Chen2014-gd} use a variational type system to allow for the
possibility of changing an expression's type, and search for an
expression whose type can be changed such that type inference would
succeed.
%
% It then attempts to deduce the error source by searching for an
% expression whose type can be changed such that type inference would
% succeed.
%
In contrast to Lerner~\etal, who search for changes at the value-level,
Chen~\etal\ search at the type-level and are thus complete due the finite
universe of types used in the program.
%
% \item \cite{chen_error-tolerant_2012}
% \item \cite{okeefe_type_1992}
% \item \cite{gomard_partial_1990}
% \item \cite{thatte_type_1988}
%

In contrast to these approaches, we do not attempt to localize or fix
the type error. Instead we try to explain it to the user using a
dynamic witness that demonstrates how the program is not just
ill-typed but truly wrong. In addition, allowing users to run their
program % (even knowing that it is wrong)
enables experimentation and the
use of debuggers to step through the program and investigate its
evolution.

\paragraph{Improving Error Messages}
The content and quality of the error messages themselves has also been
studied extensively.
%
\citet{Marceau2011-ok,Marceau2011-cy} study the effectiveness of error
messages in novice environments and present suggestions for improving
their quality and consistency.
%
\citet{Hage2006-hc} identify a variety of general heuristics to improve
the quality of type error messages, based on their teaching experience.
%
\citet{Heeren2003-db}, \citet{Christiansen2014-qc}, and
\citet{Serrano2016-oo} provide methods for library authors to specialize
type errors with domain-specific knowledge.
%
The difference with our work is more pronounced here as we do not
attempt to improve the quality of the error message, instead we search
for a witness to the error and explain it with the resulting execution
trace.

\paragraph{Running Ill-Typed Programs}
\label{sec:running-ill-typed}
\citet{Vytiniotis2012-gh} extend the \haskell
compiler GHC to support compiling ill-typed programs, but their intent
is rather different from ours. Their goal was to allow programmers to
incrementally test refactorings, which often cause type errors in
distant functions. They replace any expression that fails to type
check with a \emph{runtime} error, but do not check types
at runtime.
%
\citet{Bayne2011-cn} also provide a semantics for running
ill-typed (\java) programs, but in constrast transform the program to
perform nearly all type checking at run-time. The key difference between
Bayne~\etal\ and our work is that we use the dynamic semantics to
automatically search for a witness to the type error, while their focus
is on incremental, programmer-driven testing.

\paragraph{Testing}
\label{sec:testing}
\nanomaly is at its heart a test generator, and builds on a rich line of
work.
%
Our use of holes to represent unknown values is inspired by the work of
Runciman, Naylor, and Lindblad~\cite{Runciman2008-ka,Naylor2007-mi,Lindblad2007-oy},
%
who use lazy evaluation to drastically reduce the search space for
exhaustive test generation, by grouping together equivalent inputs by
the set of values they force. An exhaustive search is complete (up to
the depth bound), if a witness exists it will be found, but due to the
exponential blowup in the search space the depth bound can be quite
limited without advanced grouping and filtering techniques.
%
Our search is not exhaustive; instead we use random generation to fill
in holes on demand.
%
Random test generation~\cite{Claessen2000-lj,Csallner2004-bf,Pacheco2007-at}
%
is by its nature incomplete, but is able to check larger inputs than
exhaustive testing as a result.

Instead of enumerating values, which may trigger the same path through
the program, one might enumerate paths. 
%
Dynamic-symbolic execution~\cite{Godefroid2005-am,Cadar2008-kg,Tillmann2008-qc}
%
combines symbolic execution (to track which path a given input triggers)
with concrete execution (to ensure failures are not spurious). The
system collects a path condition during execution, which tracks
symbolically what conditions must be met to trigger the current
path. Upon successfully completing a test run, it negates the path
condition and queries a solver for another set of inputs that satisfy
the negated path condition, \ie inputs that will not trigger the same
path. Thus, it can prune the search space much faster than techniques
based on enumerating values, but is limited by the expressiveness of the
underlying solver.

Our operational semantics is amenable to dynamic-symbolic execution, one
would just need to collect the path condition and replace our
implementation of \gensym by a call to the solver. We chose to use lazy,
random generation instead because it is efficient, and % does not incur
the overhead of an external solver, and produces high coverage for our
domain of novice programs.

% \begin{itemize}
% \item \cite{Claessen2000-lj}
% \item 
% \item \cite{Godefroid2005-am}
% \item \cite{Cadar2008-kg}
% \item \cite{Tillmann2008-qc}
% \end{itemize}

\paragraph{Program Exploration}
Perera \etal~\cite{Perera2012-dy} present a tracing semantics
for functional programs that tags values with their provenance, enabling
a form of backwards program slicing from a final value to the sequence
of reductions that produced it. Notably, they allow the user to supply a
\emph{partial value} --- containing holes --- and present a partial slice,
containing only those steps that affected the the partial value.
% This
% system is designed to answer questions of the form ``Where did this
% value come from?'' and thus is focused on backward exploration. 
Perera~\etal\ focus on backward exploration; in contrast, our
visualization supports forward \emph{and} backward exploration, though
our backward steps are more limited.
%
Specifically, we do not support selecting a value and inserting the
intermediate terms that preceded it while ignoring unrelated computation
steps. %; this would be interesting future work.

% \ES{todo: more on program exploration}

%%% Local Variables: 
%%% mode: latex
%%% TeX-master: "main"
%%% End: 
