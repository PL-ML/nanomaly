\begin{abstract}
  %
  {Static} type errors are a common stumbling block
  for newcomers to typed functional languages.
  %
  We present a {dynamic} approach to explaining type
  errors by generating counterexample witness inputs that
  illustrate how an ill-typed program goes wrong.
  %
  First, given an ill-typed function, we symbolically
  execute the body to synthesize witness values that
  make the program go wrong.
  We prove that our procedure synthesizes
  {general} witnesses in that if a witness is
  found, then for all inhabited input types,
  there exist values that can make the function go wrong.
  %
  Second, we show how to extend the above procedure to
  produce a reduction graph that can be used to
  interactively visualize and debug witness executions.
  %
  Third, we evaluate the coverage of our approach
  on two data sets comprising over 4,500 ill-typed
  student programs.
  Our technique is able to generate witnesses for
  88\% of the programs, and our reduction graph
  yields small counterexamples for 81\% of the witnesses.
  %
  Finally, we evaluate whether our witnesses help
  students understand and fix type errors, and
  find that students presented with our witnesses
  show a greater understanding of type errors
  than those presented with a standard error message.
\end{abstract}
