\section{Type-Error Witnesses}
\label{sec:searching-witness}

% Our goal is to find concrete values that demonstrate how a
% program ``goes wrong''.

% \paragraph{Problem: Which inputs are bad?}
% %
% One approach is to randomly generate input values and use
% them to execute the program until we find one that causes
% the program to go wrong. However, to see why this approach
% is naive, consider the following example:
% %
% \begin{lstlisting}
%   let f x =
%     let y = 1 + x in
%       1. +. y
% \end{lstlisting}
% %
% What \emph{types} of inputs should we test \texttt{f} with?
% Values of type \texttt{int} and \texttt{float} are fair game,
% but values of type say, \texttt{string} or \texttt{int list}
% will cause the program to go wrong in an \emph{irrelevant}
% manner.

% \paragraph{Solution:} \RJ{STOP}


% we cannot provide \emph{completely arbitrary} inputs to
% \texttt{f}. Instead, we call \texttt{f} with a \emph{hole}, written
% \ehole{}, which is a placeholder for a value whose type we have not
% yet determined. As we execute the program, we instantiate holes with
% concrete values as demanded by the primitive operations in the
% program. For example, the hole we pass to f will be instantiated to an
% int when we reach the \lstinline{1 + x} term. Thus, y will be an int as
% well, and the program will get stuck at \lstinline{1. +. y}. \ES{this
%   reads more like overview text..}
%


% \begin{itemize}
% \item how do we run ill-typed programs?
% \item for a lang like ocaml, dynamic semantics are independent of static
%   semantics, just lambda calculus. so no problem to run ill-typed
%   program
% \item but what about functions? what type of arguments should we pass? consider
%
% \begin{lstlisting}
% let f x =
%   let y = 1 + x in
%     1. +. y
% \end{lstlisting}
%
% does \texttt{f} take an int, float, string? int and float are both
% somewhat plausible, but string or anything else is ``clearly'' bogus. so
% we cannot provide \emph{completely arbitrary} inputs to
% \texttt{f}. Instead, we call \texttt{f} with a \emph{hole}, written
% \ehole{}, which is a placeholder for a value whose type we have not
% yet determined. As we execute the program, we instantiate holes with
% concrete values as demanded by the primitive operations in the
% program. For example, the hole we pass to f will be instantiated to an
% int when we reach the \lstinline{1 + x} term. Thus, y will be an int as
% well, and the program will get stuck at \lstinline{1. +. y}. \ES{this
%   reads more like overview text..}
%
% % \item values are tagged with their types, just like ``untyped'' langs
% % \item special ``hole'' value whose type is not yet known, used for function args
% % \item on-the-fly unification to determine ``correct'' type for holes
% \end{itemize}


Next, we formalize the notion of type error witnesses as follows.
%
First, we define a core calculus within which we will work~(\S~\ref{sec:syntax}).
%
Second, we develop a (non-deterministic) operational semantics
for ill-typed programs that precisely defines the notion
of a \emph{witness}~(\S~\ref{sec:semantics}).
%
Third, we formalize and prove a notion of \emph{generality} for
witnesses, which states, intuitively, that if we find a
single witness then for \emph{every possible} type
assignment there exist inputs that are guaranteed to make
the program ``go wrong''~(\S~\ref{sec:soundness}).
%
Finally, we refine the operational semantics into a
\emph{search procedure} that returns concrete (general)
witnesses for ill-typed programs~\S~(\ref{sec:search-algorithm}).
%
We have formalized and tested our semantics and generality theorem
in \textsc{PLT-Redex}~\cite{felleisen2009semantics}.

\subsection{Syntax}
\label{sec:syntax}
\begin{figure}
% \hrule width 0.48\textwidth \vspace{0.05in}
$$
\begin{array}{rrcl}
% \emphbf{Configurations} \quad
%   & c & ::=    & \triple{e}{\vsu}{\tsu} \spmid \triple{\stuck}{\vsu}{\tsu} \\[0.05in]

\emphbf{Expressions}
  & \estuck & ::= & e \spmid \stuck \\
  & e & ::=    & v \spmid x \spmid \eapp{e}{e} \spmid \eplus{e}{e}\\
  &   & \spmid & \eif{e}{e}{e} \\
  % &   & \spmid & \elet{x}{e}{e} \\
  &   & \spmid & \epair{e}{e} \spmid \epcase{e}{x}{x}{e} \\
  &   & \spmid & \enode{e}{e}{e} \spmid \eleaf \\
  &   & \spmid & \ecase{e}{e}{x}{x}{x}{e} \\[0.1in]

\emphbf{Values}
  & v  & ::= & n \spmid b \spmid \efun{x}{e} \spmid \vhole{\thole} \spmid tr \\
  & tr & ::= & \vnode{t}{v}{v}{v} \spmid \vleaf{t} \\[0.05in]

\emphbf{Integers}
  & n & ::= &  0,1,-1,\ldots \\[0.05in]

\emphbf{Booleans}
  & b & ::= &  \etrue \spmid \efalse \\[0.05in]

\emphbf{Types}
  & t & ::=     & \tbool \spmid \tint \spmid \tfun \\
  &   &  \spmid & \tprod{t}{t} \spmid \ttree{t} \spmid \thole \\[0.05in]

\emphbf{Substitutions}
  & \vsu & ::= & \emptysu \spmid \extendsu{\vsu}{\vhole{\thole}}{v} \\
  & \tsu & ::= & \emptysu \spmid \extendsu{\tsu}{\thole}{t} \\[0.1in]
% \end{array}
% $$
% % \hrule width 0.48\textwidth
% $$
% \begin{array}{rrcl}
\emphbf{Contexts}
  & C
  & ::=
  &   	 \bullet
  \spmid \eapp{C}{e}
  \spmid \eapp{v}{C} \\
  & & \spmid & \eplus{C}{e} \spmid \eplus{v}{C} \\
  & & \spmid & \eif{C}{e}{e} \\
  % & & \spmid & \elet{x}{C}{e} \\
  & & \spmid & \epair{C}{e} \spmid \epair{v}{C} \\
  & & \spmid & \epcase{C}{x}{x}{e} \\
  & & \spmid & \enode{C}{e}{e} \\
  & & \spmid & \enode{v}{C}{e} \\
  & & \spmid & \enode{v}{v}{C} \\
  & & \spmid & \ecase{C}{e}{x}{x}{x}{e} \\[0.05in]

\emphbf{Type Contexts}
  & T &::=& \bullet \spmid \ttree{T} \spmid \tprod{T}{t} \spmid \tprod{t}{T} \\[0.05in]
\end{array}
$$

% \judgementHead{Reduction}{\eval{e}{e}}

% $$
% \begin{array}{rcl}
% \eval{C[e]&}{&C[e']} \qquad \text{if}\ \eval{e}{e'} \\
% 	\eval{\eapp{c}{v}&}{& \ceval{c}{v}}\\
% \eval{\eapp{(\efun{x}{\tau_x}{e})}{e_x}&}{&e\sub{x}{e_x}}\\
% 	\eval{\elet{x}{e_x}{e}&}{&e\sub{x}{e_x}} \\
% 	\eval{\ecase{D_j\ \overline{e}}{D_i}{\overline{y_i}}{e_i}{x}&}
% 	{&e_j\sub{x}{D_j\ \overline{e}}\sub{\overline{y_j}}{\overline{e}}} \\
% \end{array}
% $$

\caption{Syntax of \lang}
\label{fig:syntax}
\end{figure}

%
Figure~\ref{fig:syntax} describes the syntax of \lang, a simple lambda
calculus with integers, booleans, pairs, and binary trees.
%
As we are specifically interested in programs that \emph{do} go wrong,
we include an explicit \stuck\ term in our syntax. We write $\estuck$ to
denote terms that may be \stuck, and $e$ to denote terms that may not be
stuck.

\paragraph{Holes}
\label{sec:holes}
%
Recall that a key challenge in our setting is to find witnesses
that are meaningful and do not arise from choosing values from
irrelevant types.
%
We solve this problem be equipping our term language with a
notion of a \emph{hole}, written \vhole{\thole}, which represents
an \emph{unconstrained} value $\ehole$ that may be replaced with
\emph{any} value of an unknown type \thole.
%
Intuitively, the type holes \thole\ can be viewed as type variables
that we will \emph{not} generalize over.
%
A \emph{normalized} value is one that is not a hole,
but which may internally contain holes.
%
For example
$\vnode{\thole}{\vhole{\thole}}{\vleaf{\thole}}{\vleaf{\thole}}$ is a
normalized value.

\paragraph{Substitutions}
%
Our semantics ensure the generality of witnesses by incrementally
\emph{refining} holes, filling in just as much information as is
needed locally to make progress (inspired by the manner in
which SmallCheck uses lazy evaluation~\cite{runciman_smallcheck_2008}).
%
We track how the holes are incrementally filled in, by using
value (resp.\ type) \emph{substitutions} $\vsu$ (resp. $\tsu$)
that map value (resp.\ type) holes to values (resp.\ types).
%
The substitutions let us ensure that we consistently instantiate each
hole with the same (partially defined) value or type, regardless of the
multiple contexts in which the hole appears.
%
This ensures we can report a concrete (and general) witness for any
(dynamically) discovered type errors.

A \emph{normalized} value substitution is one whose co-domain is
comprised of normalized values.
%
In the sequel, we will assume and ensure that all value substitutions
are normalized.
%
We ensure additionally that the co-domain of a substitution does not
refer to any elements of its domain, \ie when we extend a substitution
with a new binding we apply the substitution to itself.

% \paragraph{Resolving Holes}
% We \emph{resolve} a hole with respect to a substitution by
% transitively applying the substitution as long as it contains
% any holes that are defined in the substitution.
% %
% We write \resolve{\thole}{\tsu} to denote the the resolution
% of \thole\ with respect to \tsu.
% %
% Note that by definition, \resolve{\thole}{\tsu} does not contain
% any holes in the domain of \tsu.


\subsection{Semantics}
\label{sec:semantics}
%
Recall that our goal is to synthesize a value that demonstrates
why (and how) a function goes wrong.
%
We accomplish this by combining evaluation with type inference,
giving us a form of dynamic type inference.
% starting with an unconstrained hole and gradually refining it
% as evaluation constrains its type.
%
Each primitive evaluation step tells us more about the types of the
program values. For example, addition tells us that the addends must be
integers, and case-analysis tells us that the scrutinee must be a tree.
%
When a hole appears in such a context, we know what type it must have
in order to make progress and can fill it in with a concrete value.

The evaluation relation is parameterized by a pair of functions:
called \emph{narrow} (\forcesym) and \emph{generate} (\gensym),
that ``dynamically'' perform type-checking and hole-filling
respectively.

\paragraph{Narrowing Types} The procedure % $\force{v}{t}{\vsu}{\tsu}$,
$$
\forcesym : v \times t \times \vsu \times \tsu \rightarrow \triple{v \cup \stuck}{\vsu}{\tsu}
$$
defined in Figure~\ref{fig:narrow}, takes as input a value $v$, a type
$t$, and the current value and type substitutions, and refines $v$ to
have type $t$ by yielding a triple of either the same value and
substitutions, or yields the stuck state if no such refinement is
possible. In the case where $v$ is a hole, it first checks in the given
$\vsu$ to see if the hole has already been instantiated and, if so,
returns the existing instantiation.
%
\begin{figure*}[ht]
$$
\begin{array}{lcl}
%% \multicolumn{3}{l}{\forcesym \ :: \  (e, t) \rightarrow \pair{e}{\vsu}} \\
\forcesym                  & : & v \times t \times \vsu \times \tsu \rightarrow \triple{v \cup \stuck}{\vsu}{\tsu} \\
% \force{v}{\thole}{\vsu}{\tsu}  & \defeq & \hspace{-1ex}
% \begin{cases}
%   \triple{v}{\vsu}{\tsu'} & \mbox{if } \tsu' = \unify{\{\typeof{v}, \thole\}}{\tsu} \\
%   \triple{\stuck}{\vsu}{\tsu} & \mbox{otherwise} \\

%   % \force{v}{\subst{\tsu}{\thole}}{\vsu}{\tsu} & \mbox{if}\ \thole \in dom(\tsu) \\
%   % \triple{v}{\vsu}{\extendsu{\tsu}{\thole}{\typeof{v}}} & \mbox{otherwise} \\
% \end{cases} \\
\force{\vhole{\thole}}{t}{\vsu}{\tsu} & \defeq & \hspace{-1ex}
\begin{cases}
  \triple{v}{\vsu}{\tsu'}    & \mbox{if } v = \lookupsu{\vsu}{\vhole{\thole}},
                                         \tsu' = \unify{\{\thole, t, \typeof{v}\}}{\tsu}\\
  \triple{\stuck}{\vsu}{\tsu} & \mbox{if } v = \lookupsu{\vsu}{\vhole{\thole}} \\
  \triple{v}{\extendsu{\vsu}{\vhole{\thole}}{v}}{\tsu'} & \mbox{if}\ \tsu' = \unify{\{\thole, t\}}{\tsu}, v = \gen{t}{\tsu'}\\
\end{cases} \\
% \begin{cases}
%   \triple{\lookupsu{\vsu}{\ehole}}{\vsu}{\tsu} & \mbox{if}\ \ehole \in dom(\vsu), \hastype{\lookupsu{\vsu}{\ehole}}{t} \\
%   \triple{\stuck}{\vsu}{\tsu}                  & \mbox{if}\ \ehole \in dom(\vsu) \\
%   \triple{v}{\extendsu{\vsu}{\ehole}{v}}{\tsu}   & v = \gen{t}\\
% \end{cases} \\
 % \begin{array}{l}
    % \mathtt{if}\ i \in \vsu\ \mathtt{then}\ \triple{\vsu(i)}{\vsu}\ \mathtt{else} \\
         % \elet{v}{\gen{t}}{\triple{v}{\ehole{i} \mapsto v}}
  % \end{array} \\
\force{n}{\tint}{\vsu}{\tsu}     & \defeq & \triple{n}{\vsu}{\tsu} \\
\force{b}{\tbool}{\vsu}{\tsu}    & \defeq & \triple{b}{\vsu}{\tsu} \\
\force{\efun{x}{e}}{\tfun}{\vsu}{\tsu} & \defeq & \triple{\efun{x}{e}}{\vsu}{\tsu} \\
\force{\epair{v_1}{v_2}}{\tprod{t_1}{t_2}}{\vsu}{\tsu} & \defeq & \triple{\epair{v_1}{v_2}}{\vsu}{\tsu''}, \mbox{if}\ \tsu' = \unify{\{\typeof{v_1}, t_1\}}{\tsu}, \tsu'' = \unify{\{\typeof{v_2}, t_2\}}{\tsu'}  \\
\force{\vleaf{t_1}}{\ttree{t_2}}{\vsu}{\tsu} & \defeq & \triple{\vleaf{t_1}}{\vsu}{\tsu'}, \mbox{if}\ \tsu' = \unify{\{t_1, t_2\}}{\tsu} \\
\force{\vnode{t_1}{v_1}{v_2}{v_3}}{\ttree{t_2}}{\vsu}{\tsu} & \defeq & \triple{\vnode{t_1}{v_1}{v_2}{v_3}}{\vsu}{\tsu'}, \mbox{if}\ \tsu' = \unify{\{t_1, t_2\}}{\tsu} \\
\force{v}{t}{\vsu}{\tsu} & \defeq & \triple{\stuck}{\vsu}{\tsu}
\end{array}
$$
\caption{Narrowing values}
\label{fig:narrow}
\end{figure*}
%
%While a hole may map to a value that \emph{contains} another hole, \eg a
%lambda or a tree, it may not map \emph{directly} to another hole,
As the value substitution is normalized, in the first case of \forcesym\ we
do not need to \forcesym\ the result of the substitution, the sub-hole
will be narrowed when the context demands it.

\paragraph{Generating Values} The (non-deterministic)
$\gen{t}{\tsu}$ in Figure~\ref{fig:gen}, takes
as input a type $t$ and returns a value of that type.
%
For base types, the procedure returns an arbitrary value of
that type.
%
For functions, it returns a lambda with a \emph{new} hole
denoting the return value, and unconstrained types (denoted
by $\thole$) yield fresh holes constrained to have type
\thole (denoted by $\vhole{\thole}$).
%
When generating a $\ttree{t}$ we must take care to ensure
the resulting tree is well-typed.
%
For a polymorphic type $\ttree{\thole}$ %, eg in \recasegoodone,
we will place holes in the generated nodes; they will be lazily filled
in later, on demand.


\begin{figure}[ht]
$$
\begin{array}{lcll}
\gensym       & :   & t \times \tsu \rightarrow v \\
\gen{\thole}{\tsu}  & \defeq  & \gen{\subst{\tsu}{\thole}}{\tsu}, &  \text{if } \thole \in dom(\tsu) \\
\gen{\tint}{\tsu}   & \defeq  & n, &  \text{non-deterministic} \\
\gen{\tbool}{\tsu}  & \defeq  & b, &  \text{non-deterministic} \\
\gen{\tprod{t_1}{t_2}}{\tsu}  & \defeq  & \epair{\gen{t_1}{\tsu}}{\gen{t_2}{\tsu}} & \\ % \begin{subarray}{l} v_1 = \gen{t_1}{\tsu}\\ v_2 = \gen{t_2}{\tsu}\end{subarray} \\
\gen{\ttree{t}}{\tsu}  & \defeq  & tr, &  \text{non-deterministic} \\
\gen{\tfun}{\tsu}   & \defeq & \efun{x}{\vhole{\thole}}, &  \text{\ehole and \thole are fresh} \\
\gen{\thole}{\tsu}  & \defeq & \vhole{\thole}, & \text{\ehole is fresh} \\
\end{array}
$$
\caption{Generating values}
\label{fig:gen}
\end{figure}


\paragraph{Steps and Traces}
\begin{figure*}
\judgementHead{Evaluation}{\step{e}{\su}{e}{\su}}
\begin{gather*}
\inference[\recontext]
  {\step{e}{\su}{e_1}{\su_1}}
  {\step{C[e]}{\su}{C[e_1]}{\su_1}}
\qquad
\inference[\restuck]
  {}
  {\step{C[\stuck]}{\su}{\stuck}{\su}}
\\ \\
\inference[\replusgood]
  {\pair{n_1}{\su_2} = \force{v_1}{\tint} \\
   \pair{n_2}{\su_3} = \force{v_2}{\tint} \\ 
   n = \eplus{n_1}{n_2}}
  {\step{\eplus{v_1}{v_2}}{\su_1}{n}{\su_1;\su_2;\su_3}}
\qquad
\inference[\replusbadone]
  {\pair{\stuck}{\su_2} = \force{v_1}{\tint}}
  {\step{\eplus{v_1}{v_2}}{\su_1}{\stuck}{\su_1;\su_2}}
\\ \\
\inference[\replusbadtwo]
  {\pair{\stuck}{\su_2} = \force{v_2}{\tint}}
  {\step{\eplus{v_1}{v_2}}{\su_1}{\stuck}{\su_1;\su_2}}
\qquad
\inference[\reifgoodone]
  {\pair{\etrue}{\su_2} = \force{v}{\tbool}}
  {\step{\eif{v}{e_1}{e_2}}{\su_1}{e_1}{\su_1;\su_2}}
\\ \\
\inference[\reifgoodtwo]
  {\pair{\efalse}{\su_2} = \force{v}{\tbool}}
  {\step{\eif{v}{e_1}{e_2}}{\su_1}{e_2}{\su_1;\su_2}}
\qquad
\inference[\reifbad]
  {\pair{\stuck}{\su_2} = \force{v}{\tbool}}
  {\step{\eif{v}{e_1}{e_2}}{\su_1}{\stuck}{\su_1;\su_2}}
\\ \\
\inference[\reappgood]
  {\pair{\efun{x}{e}}{\su_2} = \force{v_1}{\tfun{\thole{}}{\thole{}}}}
  {\step{\eapp{v_1}{v_2}}{\su_1}{e\sub{x}{v_2}}{\su_1;\su_2}}
\qquad
\inference[\reappbad]
  {\pair{\stuck}{\su_2} = \force{v_1}{\tfun{\thole{}}{\thole{}}}}
  {\step{\eapp{v_1}{v_2}}{\su_1}{\stuck}{\su_1;\su_2}}
\\ \\
\inference[\relet]
  {}
  {\step{\elet{x}{v}{e}}{\su}{e\sub{x}{v}}{\su}}
\end{gather*}
\\ % [0.05in]
\relDescription{\forcesym and \gensym}
\begin{gather*}
\begin{array}{lcl}
\force{\ehole{i}}{t} & \defeq & \elet{v}{\gen{t}}{\pair{v}{\ehole{i} \mapsto v}} \\
\force{v}{\ehole{}}  & \defeq & \pair{v}{\emptysu} \\
\force{n}{\tint}    & \defeq & \pair{n}{\emptysu} \\
\force{v}{\tint}    & \defeq & \pair{\stuck}{\emptysu} \\
\force{b}{\tbool}   & \defeq & \pair{b}{\emptysu} \\
\force{v}{\tbool}   & \defeq & \pair{\stuck}{\emptysu} \\
\force{\efun{x}{e}}{\tfun{\thole{}}{\thole{}}} & \defeq & \pair{\efun{x}{e}}{\emptysu} \\
\force{v}{\tfun{\thole{}}{\thole{}}} & \defeq & \pair{\stuck}{\emptysu} \\
\end{array}
\qquad
\begin{array}{lcll}
\gen{\tint}   & \defeq & n & \\
\gen{\tbool}  & \defeq & b & \\
\gen{\tfun{t_1}{t_2}} & \defeq & \efun{x}{\ehole{i}}, & \quad \text{$i$ is fresh} \\
\gen{\thole{}} & \defeq & \ehole{i}, & \quad \text{$i$ is fresh} \\
\end{array}
\end{gather*}
\caption{Evaluation relation}
\label{fig:operational}
\end{figure*}

%
% WRW notes that Figure 4 does not seem to handle recursion (it's not clear
% how the let rule would work for something "let rec"-y, and there's not
% function call rule). I only mention this because I can imagine a reviewer
% wondering about your ability to generate good witnesses for function
% types. This could likely be addressed in text, by a forward reference to
% Section 3.4 where higher-order functions are handled, without changing
% any of the formalisms at the last minute.
Figure~\ref{fig:operational} describes the small-step contextual
reduction semantics for \lang.
%
A configuration is a triple $\triple{\estuck}{\vsu}{\tsu}$ of an
expression $e$ or the stuck term $\stuck$, a value substitution $\vsu$,
and a type substitution $\tsu$.
%
% We write \estuck to
% denote either an expression $e$ or \stuck.
%
We write $\step{\estuck}{\vsu}{\tsu}{\estuck'}{\vsu'}{\tsu'}$ if the state
$\triple{\estuck}{\vsu}{\tsu}$ transitions in a \emph{single step} to
$\triple{\estuck'}{\vsu'}{\tsu'}$.
%
A (finite) \emph{trace} $\trace$ is a sequence of configurations
$\triple{\estuck_0}{\vsu_0}{\tsu_0}, \ldots, \triple{\estuck_n}{\vsu_n}{\tsu_n}$ such that
$\forall 0 \leq i < n$, we have
$\step{\estuck_i}{\vsu_i}{\tsu_i}{\estuck_{i+1}}{\vsu_{i+1}}{\tsu_{i+1}}$.
%
We write \steptr{\trace}{\estuck}{\vsu}{\tsu}{\estuck'}{\vsu'}{\tsu'} if $\trace$ is
a trace of the form $\triple{\estuck}{\vsu}{\tsu},\ldots,\triple{\estuck'}{\vsu'}{\tsu'}$.
%
We write \steps{\estuck}{\vsu}{\tsu}{\estuck'}{\vsu'}{\tsu'} if
\steptr{\trace}{\estuck}{\vsu}{\tsu}{\estuck'}{\vsu'}{\tsu'} for some trace $\trace$.

\paragraph{Primitive Reductions}
%
% \RJ{Put high-level intuition about how "dynamic HM" is
% formalized in op-sem, to set up next few paragraphs}
%
Primitive reduction steps --- addition, if-elimination,
function application, and data construction and case
analysis --- use \forcesym to ensure that values have
the appropriate type (and that holes are instantiated)
before continuing the computation.
%
Importantly, beta-reduction \emph{does not} type-check its
argument, it only ensures that ``the caller'' $v_1$ is indeed
a function.

\paragraph{Recursion}
%
% \RJ{this para appears out of nowhere --- non-sequitur}
Our semantics lacks a built-in $\mathtt{fix}$ construct
for defining recursive function, which may surprise
the reader.
%
Fixed-point operators often cannot be typed in static type
systems, but our system would simply approximate its type
as $\tfun$, apply it, and move along with evaluation.
%
Thus we do not need a built-in recursion construct, we
can use any of the standard fixed-point operators.

%  but we are not concerned with \emph{assigning}
% types to terms, rather with showing that \emph{no type}
% can be assigned.
% %
% We are simply executing the untyped $\lambda$-calculus,
% which has no issue handling recursion.

%% \begin{thm}
%% \label{thm:all-reduce}
%%   Every closed expression $e$ reduces to a value $v$ (which may be \stuck).
%% \ES{do we really need to state this, or is it obvious?}
%% \end{thm}

% \begin{proof}%[Proof of \autoref{thm:all-reduce}]
%   Simple induction on the evaluation relation.
% \end{proof}

\subsection{Generality}\label{sec:soundness}

A key technical challenge in generating witnesses is
that we have no (static) type information to rely upon.
%
Thus, we must avoid the trap of generating \emph{spurious}
witnesses that arise from picking irrelevant values, when
instead there exist perfectly good values of a \emph{different}
type, under which the program would not have gone wrong.
%
We now show that our evaluation relation instantiates holes
in a \emph{general} manner. That is, given a lambda-term $f$,
if we have $\steps{\eapp{f}{\vhole{\thole}}}{\emptysu}{\emptysu}{\stuck}{\vsu}{\tsu}$,
then \emph{for every} concrete type $t$, we can find a value
$v$ of type $t$ such that $\eapp{f}{v}$ goes wrong.

\begin{thm}{\textbf{[Witness Generality]}}
\label{thm:soundness}
  For any lambda-term $f$, if
  \hbox{$\steptr{\trace}{\eapp{f}{\vhole{\thole}}}{\emptysu}{\emptysu}{\stuck}{\vsu}{\tsu}$,}
  then for every % inhabitable
  type
  % \footnote{We exclude builtin functions that subvert the type system, \eg \texttt{Obj.magic}, and thus consider the type $\forall a b. \tfun{a}{b}$ to be uninhabitable.}
  $t$ there exists a $v$ of type $t$ such that
  $\steps{\eapp{f}{v}}{\emptysu}{\emptysu}{\stuck}{\vsu}{\tsu}$.
\end{thm}

We need to develop some machinery in order to prove this theorem.
First, we show how our evaluation rules encode a dynamic form of
type inference, and next, we show that the types inferred via
evaluation are indeed maximally general.

\paragraph{The Type of a Value} The \emph{dynamic type}
of a value $v$ is defined as a function $\typeof{v}$ shown
in Figure~\ref{fig:typeof}.
%
The types of primitive values are defined in the natural manner.
%
The types of functions are \emph{approximated}, which is all
that is needed to ensure an application does not get stuck, \eg
$$\typeof{\efun{x}{\eplus{x}{1}}} = \tfun$$
instead of $\tint \rightarrow \tint$.
%
The types of (polymorphic) data types are obtained from the
labels on their values.

\begin{figure}[ht]
\[ \begin{array}{lcll}
    \typeof{n}   & \defeq & \tint & \\
    \typeof{b}   & \defeq & \tbool & \\
    \typeof{\efun{x}{e}} & \defeq & \tfun \\
    \typeof{\epair{v_1}{v_2}} & \defeq & \tprod{\typeof{v_1}}{\typeof{v_2}} \\
    \typeof{\vleaf{t}} & \defeq & \ttree{t} \\
    \typeof{\vnode{t}{v_1}{v_2}{v_3}} & \defeq & \ttree{t} \\
    \typeof{\vhole{\thole}} & \defeq & \thole \\
    % \typeof{\eleaf} & \defeq & \ttree{\thole}, & \quad \text{\thole is fresh} \\
    % \typeof{\enode{v_1}{v_2}{v_3}} & \defeq & \ttree{\thole}, & \quad \text{\thole is fresh} \\
    % \typeof{e} & \defeq & \thole, & \quad \text{\thole is fresh} \\
  \end{array} \]
\caption{The \emph{dynamic type} of a value.}
\label{fig:typeof}
\end{figure}

\paragraph{Dynamic Type Inference}
We can think of the evaluation of \eapp{f}{\vhole{\thole}}
as synthesizing a partial instantiation of \thole, and hence,
\emph{dynamically inferring} a (partial) type for $f$'s input.
%
We can extract this type from an evaluation trace by
applying the final type substitution to \thole.
% \emph{resolving} the \thole\ with the final type
% substitution at the end of the trace.
%
Formally, we say that if
$\steptr{\trace}{\eapp{f}{\vhole{\thole}}}{\emptysu}{\emptysu}{\estuck}{\vsu}{\tsu}$,
then the \emph{partial input type} of $f$ up to $\trace$, written
\ptype{\trace}{f}, is $\resolve{\thole}{\tsu}$.

%repeatedly
%applying the final type substitution to \thole until it contains no
%holes in the domain of the substitution. We will call this process of
%repeated substition \emph{resolving} a hole, and will use
%\resolve{\thole}{\tsu} to denote the the resolution of \thole with
%respect to \tsu.



%$\typeof{\subst{\vsu}{\ehole}}$.
% \ES{should we say $\subst{\tsu}{\thole}$ instead? would need a helper function that does the repeated application of \tsu\ until the result has no $\thole \in dom(\tsu)$}
%
%We will omit the subscript when we wish to refer to the final partial
%type, \ie\ at the step where the expression has been reduced to a value
%(or stuck.)

% \begin{lem}
% \label{lem:narrow-tsu}
% If $\trace \defeq \triple{\eapp{f}{\vhole{\thole}}}{\emptysu}{\emptysu},\ldots$
% and $\trace' \defeq \trace, \triple{\estuck'}{\vsu'}{\tsu'}$
%     (\ie $\trace'$ is a single-step extension of $\trace$)
% and $\tsu \neq \tsu'$
% then the final step must have \emph{successfully} invoked \forcesym.
% \end{lem}

% \paragraph{Narrowing}
% %
% Only a successful call to \forcesym can change the partial
% input type of $f$.
% %
% \begin{lem}
% \label{lem:force-inst}
% If
% $\trace \defeq \triple{\eapp{f}{\vhole{\thole}}}{\emptysu}{\emptysu},\ldots,\triple{e}{\vsu}{\tsu}$
% and
% $\trace' \defeq \trace, \step{e}{\vsu}{\tsu}{e'}{\vsu'}{\tsu'}$
% (\ie $\trace'$ is a single-step extension of $\trace$)
% and
% $\ptype{\trace}{f} \neq \ptype{\trace'}{f}$,
% then the final step $\step{e}{\vsu}{\tsu}{e'}{\vsu'}{\tsu'}$
%  successfully invokes \forcesym.
% %
% \ES{not clear we still need this}
% \end{lem}

% \begin{proof}
%   By case analysis on the evaluation rules.
%   %
%   If $\ptype{\trace}{f} \neq \ptype{\trace'}{f}$ then,
%   % one of the holes in $f$'s
%   % argument must have been instantiated with a concrete value at the last step.
%   by the definition of $\ptype{\trace}{f}$, $\tsu \neq \tsu'$, as \thole
%   does not change.
%   %
%   % An examination of the rules shows that only place this happens is
%   % in the second case of \forcesym.
%   An examination of the rules shows that only \forcesym can update \tsu,
%   and furthermore that only the successful cases of \forcesym do update
%   \tsu.
% \end{proof}


\paragraph{Compatibility}
%
A type $s$ is \emph{compatible} with a type $t$, written \tcompat{s}{t},
if $\exists \tsu.\ \subst{\tsu}{s} = \subst{\tsu}{t}$.
%
That is, two types are compatible if there exists a type substitution
that maps both types to the same type.
%
A value $v$ is \emph{compatible} with a type $t$, written \vcompat{v}{t},
if $\tcompat{\typeof{v}}{t}$, that is, if the dynamic type of $v$ is
compatible with $t$.

% \paragraph{Preservation}
% We prove that each evaluation step \emph{refines} the partial input type
% of $f$, \ie\ preserves type compatibility.
% %
% \begin{lem}
% \label{lem:refine-partial}
% If $\trace \defeq \triple{\eapp{f}{\vhole{\thole}}}{\emptysu}{\emptysu},\ldots$ and
% $\trace'$ is a single-step extension of $\trace$, % \defeq \trace, \triple{\estuck'}{\vsu'}{\tsu'}$
% %
% %The partial type of $f$ upto $\trace$ is compatible
% %with the partial type upto $\trace'$, \ie\
% %
% then \tcompat{\ptype{\trace}{f}}{\ptype{\trace'}{f}}.
% %
% \ES{not clear we still need this}
% \end{lem}
% \begin{proof}
%   By case analysis on the evaluation rules.
%   %
%   First note that by Lemma~\ref{lem:force-inst} we can immediately
%   discharge the \rulename{E-*-Bad} rules as they cannot change
%   \ptype{\trace}{f} at all, and are thus trivially
%   compatibility-preserving. For the \rulename{E-*-Good} rules we can
%   show that, by virtue of \forcesym succeeding, all must preseve
%   compatibility.
%   % Note that all rules preserve partial types with the exception of when
%   % \forcesym\ is called on a hole, in which case we may instantiate the hole with
%   % a concrete value.
%   % %
%   % But $\typeof{\vhole{\thole}} = \thole$, which is compatible with any type.
% \end{proof}

\paragraph{Type Refinement}
A type $s$ is a \emph{refinement} of a type $t$, written $\tsub{s}{t}$,
if $\exists \theta. s = \subst{\theta}{t}$.
%
In other words, $s$ is a refinement of $t$ if there exists a type
substitution that maps $t$ directly to $s$.
%
A type $t$ is a \emph{refinement} of a value $v$, written $\tsub{t}{v}$,
if $\tsub{t}{\typeof{v}}$, \ie if $t$ is a refinement of the
dynamic type of $v$.
%
\ES{calling a type a refinement of a value sounds really weird..}

We prove that the partial instantiation of a type hole $\thole$
is always a refinement of the partial instantiation of the
associated value hole $\vhole{\thole}$.

\paragraph{Preservation}
%
We prove two preservation lemmas. First, we show that each evaluation
step refines the partial input type of $f$, thus preserving type
compatibility.
%
\begin{lem}
\label{lem:vsu-ext}
If  $\trace \defeq \triple{\eapp{f}{\vhole{\thole}}}{\emptysu}{\emptysu},\ldots$
and $\trace'$ is a single-step extension of $\trace$
and $\ptype{\trace}{f} \neq \ptype{\trace'}{f}$
% and $\resolve{\thole}{\tsu} \neq \resolve{\thole}{\tsu'}$
then $\tsu' = \tsu[\thole_1 \mapsto t_1] \ldots [\thole_n \mapsto t_n]$.
\end{lem}
\begin{proof}
  By case analysis on the evaluation rules.
  %
  $\thole$ does not change, so if the partial input types differ then
  $\tsu \neq \tsu'$.
  %
  Note that only \forcesym can change \tsu, and it can only do so
  via \unifysym, which can only extend \tsu.
  % and only the first case,
  % when we extend \vsu\ with $[\ehole \mapsto v]$,
  % does.
  %
  % \ES{fill in cases in the appendix}
\end{proof}
%
Second, we show that at each step of evaluation, the partial input type of $f$
is a refinement of the instantiation of $\vhole{\thole}$.
%
\begin{lem}
\label{lem:resolve-compat}
For all traces
$\trace \defeq \triple{\eapp{f}{\vhole{\thole}}}{\emptysu}{\emptysu},\ldots,\triple{e}{\vsu}{\tsu}$,
$\vsub{\ptype{\trace}{f}}{\resolve{\vhole{\thole}}{\vsu}}$.
\end{lem}
\begin{proof}
  By induction on $\trace$.
  %
  In the base case $\trace = \triple{\eapp{f}{\vhole{\thole}}}{\emptysu}{\emptysu}$
  and $\thole$ is trivially a refinement of $\vhole{\thole}$.
  %
  In the inductive case, consider the single-step extension of $\trace$,
  $\trace' = \trace,\triple{e'}{\vsu'}{\tsu'}$.
  %
  We show by case analysis on the evaluation rules that if
  $\vsub{\ptype{\trace}{f}}{\resolve{\vhole{\thole}}{\vsu}}$, then
  $\vsub{\ptype{\trace'}{f}}{\resolve{\vhole{\thole}}{\vsu'}}$.
  %
\end{proof}

\paragraph{Incompatible Types Are Wrong}
\emph{For all} types that are \emph{incompatible} with the
partial input type up to trace $\trace$, there exists a value
that will cause $f$ to get stuck in \emph{at most} $k$ steps,
where $k$ is the length of $\trace$.

\begin{lem}
\label{lem:k-stuck}
For all types $t$,
if $\steptr{\trace}{\eapp{f}{\vhole{\thole}}}{\emptysu}{\emptysu}{e}{\vsu}{\tsu}$ and
   $\tincompat{t}{\ptype{\trace}{f}}$,
   then there exists a $v$ such that $\hastype{v}{t}$ and
   $\steps{\eapp{f}{v}}{\emptysu}{\emptysu}{\stuck}{\vsu'}{\tsu'}$ in at most
   $k$ steps, where $k$ is the length of $\trace$.
\end{lem}
\begin{proof}
  We can construct $v$ from $\trace$ as follows.
  %
  Let
  $$
  \trace_i = \triple{\eapp{f}{\vhole{\thole}}}{\emptysu}{\emptysu},
             \ldots,
             \triple{e_{i-1}}{\vsu_{i-1}}{\tsu_{i-1}},
             \triple{e_{i}}{\vsu_{i}}{\tsu_{i}}
  $$
  be the shortest prefix of $\trace$ such that
  $\tincompat{\ptype{\trace_i}{f}}{t}$.
  %
  We will show that $\ptype{\trace_{i-1}}{f}$ % $\resolve{\thole}{\tsu_{i-1}}$
  must contain some other hole $\thole'$ that is
  instantiated at step $i$.
  %
  Furthermore, $\thole'$ is instantiated in such a way that
  $\tincompat{\ptype{\trace_i}{f}}{t}$.
  % $\tincompat{\resolve{\thole}{\tsu_{i}}}{t}$.
  %
  Finally, we will show that if we had instantiated $\thole'$ such that
  $\tcompat{\ptype{\trace_i}{f}}{t}$,
  % $\tcompat{\resolve{\thole}{\tsu_{i}}}{t}$,
  the current step would have gotten $\stuck$.

  % Next, we introduce a notion of type % and value
  % contexts, defined analogously to evaluation contexts.
  % %
  % $$
  % \begin{array}{lcl}
  %   T &::=& \bullet \spmid \ttree{T} \spmid \tprod{T}{t} \spmid \tprod{t}{T} \\
  %   % V &::=& \bullet \spmid \vnode{t}{V}{v}{v} \spmid \vnode{t}{v}{V}{v} \spmid \vnode{t}{v}{v}{V} \\
  % \end{array}
  % $$

  % By Lemma~\ref{lem:fixme} we know that
  % $\vcompat{\resolve{\vhole{\thole}}{\vsu}}{\resolve{\thole}{\tsu}}$.
  %
  By Lemma~\ref{lem:vsu-ext} we know that
  $\tsu_{i} = \tsu_{i-1}[\thole_1 \mapsto t_1] \ldots [\thole_n \mapsto t_n]$.
  %
  We will assume, without loss of generality, that
  $\tsu_{i} = \tsu_{i-1}[\thole' \mapsto t']$.
  %
  Since $\tsu_{i-1}$ and $\tsu_{i}$ differ only in $\thole'$ but the resolved
  types differ,
  $\ptype{\trace_{i-1}}{f} = \intctx{\thole'}$
  and
  $\ptype{\trace_i}{f} = \intctx{t'}$.
  % $\resolve{\thole}{\tsu_{i-1}} = \intctx{\thole'}$
  % and
  % $\resolve{\thole}{\tsu_{i}} = \intctx{t'}$.
  %
  By Lemma~\ref{lem:context-compat},
  let $s$ be a concrete type such that $\intctx{s} = t$.
  %
  We show by case analysis on the evaluation rules that
  %
  $$\step{e_{i-1}}{\vsu_{i-1}}{\tsu_{i-1}[\thole' \mapsto s]}{\stuck}{\vsu}{\tsu}$$

  Finally, by Lemma~\ref{lem:resolve-compat} we know that
  % $\vsub{\resolve{\thole}{\tsu_{i-1}}}{\resolve{\ehole}{\vsu_{i-1}}}$,
  $\vsub{\ptype{\trace_{i-1}}{f}}{\resolve{\vhole{\thole}}{\vsu_{i-1}}}$
  and thus $\vsub{\intctx{\thole'}}{\resolve{\vhole{\thole}}{\vsu_{i-1}}}$.
  %
  Let
  $$
  \begin{array}{lcl}
  u &=& \gen{s}{\tsu} \\
  v &=& \resolve{\vhole{\thole}}{\vsu_{i-1}}\sub{\ehole'[\thole']}{u}\sub{\thole'}{s} \\
  \end{array}
  $$
  $\steps{\eapp{f}{v}}{\emptysu}{\emptysu}{\stuck}{\vsu}{\tsu}$ in $i$ steps.
  %and
  % $\vsub{\resolve{\thole}{\tsu_{i}}}{\resolve{\ehole}{\vsu_{i}}}$.
  % \vcompat{\resolve{\vhole{\thole}}{\vsu_{i-1}}}{\resolve{\thole}{\tsu_{i-1}}}
  % =
  % \tcompat{\ptype{i-1}{f}}{t}
  % \\
  % \vcompat{\resolve{\vhole{\thole}}{\vsu_{i}}}{\resolve{\thole}{\tsu_{i}}}
  % =
  % \tincompat{\ptype{i}{f}}{t}
  %
  % Finally, by the definition of $\typeofsym$,
  % we know that $\resolve{\vhole{\thole}}{\vsu_{i-1}} = \invctx{v}$
  % such that $\vsub{T'[\thole']}{v}$.
  %such that $\resolve{\typeof{v}}{\tsu_{i-1}} = T'[\thole']$
  %
  %\ES{need to connect \ptype{\trace{i}}{f} and \resolve{\vhole{\thole}}{\vsu_i}}
  %
  % Now, consider any value $u$ such that $\vcompat{\invctx{u}}{t}$,
  % we show by case analysis on the evaluation relation that
  % $\invctx{u}$ could not make progress at this step, and thus
  % $\steps{\eapp{f}{\invctx{u}}}{\emptysu}{\emptysu}{\stuck}{\vsu'}{\tsu'}$.
\end{proof}

% \paragraph{Incompatible Values Are Wrong}
%
% \emph{Any} value that is \emph{incompatible} with
% the partial input type upto trace $\trace$ will
% cause $f$ to get stuck in \emph{at most} $k$
% steps, where $k$ is the length of $\trace$.
% %
% \begin{lem}
% \label{lem:k-stuck}
%   For all $v$,
%   if \steptr{\trace}{\eapp{f}{\vhole{\thole}}}{\emptysu}{\emptysu}{e}{\vsu}{\tsu} and
%      \vincompat{v}{\ptype{\trace}{f}},
%   then
%      \steps{\eapp{f}{v}}{\emptysu}{\emptysu}{\stuck}{\vsu}{\tsu}
%      in at most $k$ steps, where $k$ is the length of $\trace$.
% \end{lem}
% \begin{proof}
% By induction on $k$, the length of $\trace$.
% %
% Suppose {\vincompat{v}{\ptype{\trace}{f}}}.
% %
% We show that \steps{\eapp{f}{v}}{\emptysu}{\emptysu}{\stuck}{\vsu}{\tsu}
% in at most $k$ steps.
% %
% The base case, $k = 0$ is trivial, $\trace$ is empty
% and so $\ptype{\trace}{f}$ is a hole that is compatible
% with \emph{every} value $v$.
% %
% In the inductive case, let
% $\trace' = \triple{\eapp{f}{\vhole{\thole}}}{\emptysu}{\emptysu},\ldots,\triple{e'}{\vsu'}{\tsu'}$
% be the prefix of $\trace$ of length $k-1$.
% %
% Furthermore, let $s_{\trace} = \ptype{\trace}{f}$ and $s_{\trace'} = \ptype{\trace'}{f}$.
% %
% Let us split cases on whether $v$ is compatible with $s_{\trace'}$.
% %
% \begin{description}
% \item [Case \vincompat{v}{s_{\trace'}}:]
%   The inductive hypothesis applies.

% \item [Case $\vcompat{v}{s_{\trace'}}$ but $\vincompat{v}{s_{\trace}}$:]
%   Since $\vcompat{v}{s_{\trace'}}$ but $\vincompat{v}{s_{\trace}}$ we know
%   that $s_{\trace'} \neq s_{\trace}$.
%   By Lemma~\ref{lem:force-inst} we know that we must have
%   invoked \forcesym\ at step $k$.
%   %
%   The Preservation Lemma~\ref{lem:refine-partial} implies that
%   $\tcompat{s_{\trace'}}{s_{\trace}}$, which means we must have
%   specifically narrowed $s_{\trace'}$ to a type incompatible with $v$.
%   %
%   A case analysis of the evaluation rules shows that such an
%   invocation of \forcesym\ at step $k$
%   cannot succeed, \ie\ yields \stuck.
%    \RJ{some intuition needed --- seems like key step.}
% \end{description}
% \end{proof}

\begin{proof}[\textbf{Proof of Theorem~\ref{thm:soundness}}]
%
Suppose $\trace$ witnesses that $f$ gets stuck,
and let $s = \ptype{\trace}{f}$.
We show that \emph{all} types $t$ have stuck-inducing
values by splitting cases on whether $t$ is
compatible with $s$. %the partial type upto $\trace$.
%
\begin{description}
\item [Case \tcompat{s}{t}:]
  Let $\trace = \triple{\eapp{f}{\vhole{\thole}}}{\emptysu}{\emptysu},\ldots,\triple{\stuck}{\vsu}{\tsu}$.
  %
  The value $v = \resolve{\vhole{\thole}}{\vsu}$ demonstrates that
  $\eapp{f}{v}$ gets stuck.
\item [Case \tincompat{s}{t}:] By Lemma~\ref{lem:k-stuck}, we can derive
  a $v$ from $\trace$ such that $\hastype{v}{t}$ and $\eapp{f}{v}$ gets
  stuck.
% \item [Case \tincompat{s}{t}:] By Lemma~\ref{lem:k-stuck}, every $v$
%   such that \hastype{v}{t} demonstrates that $\eapp{f}{v}$ gets stuck.
%   % \ES{do we need to say anythign else?}
\end{description}
\end{proof}

\subsection{Search Algorithm}
\label{sec:search-algorithm}
%
So far, we have seen how a trace leading to a stuck configuration yields
a general witness demonstrating that the program is ill-typed (\ie\ goes
wrong for at least one input of every type.)
In particular, we have shown how to non-deterministically find a witnesses
for a function of a \emph{single} argument.

In order to convert the semantics into a \emph{procedure} for finding
witnesses, we must address two challenges.
%
First, we must resolve the non-determinism introduced by \gensym.
%
Second, in the presence of higher-order functions and currying,
we must determine how many concrete values to generate to make
execution go wrong (as we cannot rely upon static typing to
provide this information.)

The witness generation procedure $\genWitnessN$ is formalized in
Figure~\ref{fig:algo-gen-witness}.
%
Next, we describe its input and output, and how it
addresses the above challenges to search the space of possible
executions for general type error witnesses.

\paragraph{Inputs and Outputs}
%
The problem of generating inputs is undecidable in general.
%
Our witness generation procedure takes two inputs.
%
First, a search bound $k$ which is used to define the \emph{number} of
traces to explore. (We assume, without loss of generality, that all
traces are finite.)
%
Second, the target expression $e$ that contains the type error,
and may be a curried function of multiple arguments.
%
The witness generation procedure returns as output a list of (general)
witness expressions, each of which is of the form $e\ v_1 \ldots v_n$.
%
The \emph{empty} list is returned when no witness can be found after
exploring $n$ traces.


\paragraph{Modeling Semantics}
%
We resolve the non-determinism in the operational semantics
(\S~\ref{sec:semantics}) via the procedure
%
$$
\evalN : e \rightarrow \triple{v \cup \stuck}{\vsu}{\tsu}^{*}
$$
%
Due to the non-determinism introduced by \gensym, a call
$\evalfn{e}$ returns a \emph{list}
of possible results of the form $\triple{v \cup \stuck}{\vsu}{\tsu}$
such that $\steps{e}{\emptysu}{\emptysu}{v \cup \stuck}{\vsu}{\tsu}$.

\paragraph{Currying}
We address the issue of currying by defining a procedure \genArgs{e},
defined in Figure~\ref{fig:algo-gen-args}, that takes as input an
expression $e$ and produces a \emph{saturated} expression of the form
$\eapp{e}{\ehole_1[\thole_1] \ldots \ehole_n[\thole_n]}$ that
\emph{does not} evaluate to a lambda.
%
This is achieved with a simple loop that keeps adding holes to the
target application until evaluating the term yields a non-lambda value.
%
% The helper functions @witness@ and @close@ are used to respectively
% apply the parameters to the target and close the result under the top-level
% let-binders, prior to invoking \hbox{@eval@.}
%
% That is, @mkApps@ creates a nested sequence of applications in
% the usual left-associative style, and @mkLets@ takes a list of
% binders and a body expression, and creates a sequence of nested
% let-binders that close the body expression.

\begin{figure}[t]
% $$
% \begin{array}{lcl}
% \genArgs{e} &\defeq& \begin{cases} \genArgs{\addArg{e}},& \mbox{if } \triple{\efun{x}{e'}}{\vsu}{\tsu},\ldots = \doeval{e} \\
%  e,& \mbox{otherwise} \\
% \end{cases}
% \\
% \addArg{e} &\defeq& \eapp{e}{\vhole{\thole}}, \qquad \mbox{where } \ehole, \thole \mbox{ are fresh }
% \end{array}
% $$
$$
\begin{array}{lclr}
\genArgsN   & : & e \rightarrow e \\
\genArgs{e} & = & \mbox{\textbf{case }} \evalfn{e} \mbox{\textbf{ of}} \\
 \quad \triple{\efun{x}{e}}{\vsu}{\tsu},\ldots &\rightarrow& \genArgs{\eapp{e}{\vhole{\thole}}} & (\ehole, \thole \mbox{ are fresh}) \\
 \quad \_ &\rightarrow& e \\
\end{array}
$$
\caption{Generating a saturated application.}
\label{fig:algo-gen-args}

\end{figure}

\paragraph{Generating Witnesses}
%
Finally, Figure~\ref{fig:algo-gen-witness} summarizes the overall
implementation of our search for witnesses, in the procedure
$\genWitness{k}{e}$, which takes as input a bound $k$ and the
target expression $e$, and returns a list of witness expressions
$\eapp{e}{v_1 \ldots v_n}$ that demonstrate how the input program
gets stuck.
%
The search proceeds as follows.
%
\begin{enumerate}
  \item We invoke $\genArgs{e}$ to produce a \emph{saturated}
        application $e_{sat}$.

  \item We take the first $k$ traces returned by $\evalN$
        on the target $e_{sat}$, and

  \item We extract the substitutions corresponding to the
        $\stuck$ traces, and use them to return the list
        of witnesses.
\end{enumerate}
%
We obtain the following corollary of Theorem~\ref{thm:soundness}:

\begin{cor}{\textbf{[Witness Generation]}}
\label{thm:generation}
  If $$\genWitness{k}{e} = \triple{\eapp{e}{v_1 \ldots v_n}}{\vsu}{\tsu}\ldots$$
  then for all types $t_1 \ldots t_n$ there exist values $w_1 \ldots w_n$ such that
  $\steps{\eapp{e}{w_1 \ldots w_n}}{\emptysu}{\emptysu}{\stuck}{\vsu'}{\tsu'}$.
\end{cor}
\begin{proof}
  For any function $f$ of multiple arguments, we can define $f'$ to be
  the uncurried version of $f$ that takes all of its arguments as a
  nested pair, and then apply Theorem~\ref{thm:soundness} to $f'$.
\end{proof}

\begin{figure}[t]
\centering
$$
\begin{array}{lclr}
\genWitnessN       & : & \mathsf{Nat} \times e \rightarrow e^{*} & \\
\genWitness{n}{e}  & = & \{ \resolve{e_{sat}}{\vsu} \mid \vsu \in \Sigma \} & \\
\quad \mbox{\textbf{where}} &    & & \\
\quad \quad e_{sat} & = & \genArgs{e} & (1) \\
\quad \quad res    & = & \takefn{n}{\evalfn{e_{sat}}} & (2) \\
\quad \quad \Sigma & = & \{ \vsu\ \mid \triple{\stuck}{\vsu}{\tsu} \in res\} & (3)
\end{array}
$$
%% \begin{mcode}
%% $\genWitnessN :: (\mathsf{Nat} \times e) \rightarrow 2^{e}$
%% $\genWitness{n}{e} = \{ \resolve{\eapp{e}{vs}}{\vsu} \mid \vsu \in \Sigma \}$
  %% where
   %% $vs     = \genArgs{e}$               -- (1)
   %% $res    = \takefn{n}{\evalfn{\triple{\eapp{e}{vs}}{\emptysu}{\emptysu}}}$  -- (2)
   %% $\Sigma = \{ \vsu\ \mid \triple{\stuck}{\vsu}{\tsu} \in res\}$   -- (3)
%% \end{mcode}
\caption{Generating witnesses.}
\label{fig:algo-gen-witness}
\end{figure}

%% \begin{figure}[t]
  %% \centering
  %% \begin{mcode}
  %% -- transitive small-step reduction, returning a list of results
  %% eval :: ($e$, $\su$) -> [($v$, $\su$)]
%%
  %% -- | is a value stuck?
  %% isStuck :: $v$ -> Bool
%%
  %% mkApps  :: $e$ -> [$e$] -> $e$
  %% mkLets  :: [($x$, $e$)] -> $e$ -> $e$
  %% \end{mcode}
  %% \caption{Expression API}
  %% \label{fig:expression-api}
%% \end{figure}
%% %
%% We also define a few helper functions for manipulating expressions:
%% \begin{itemize}
%% \item @subst@ applies a substitution of holes to a value,
%% \item @mkApps@ creates a nested sequence of applications in the usual
  %% left-associative style,
%% \item @mkLets@ takes a list of binders and a body expression, and
  %% creates a sequence of nested let-binders, and
%% \item @isStuck@ tests whether a value is the \stuck term.
%% \end{itemize}

%%%  \begin{figure*}[t]
  %%%  \centering
  %%%  \begin{mcode}
  %%%  check :: [($x$, $e$)] -> Result
  %%%  check bnds =
    %%%  -- (2) search for a witness
    %%%  case find (isStuck . fst) results of
      %%%  Nothing      -> Safe
      %%%  Just (_, su) -> Unsafe (mkApps f (subst su args))
    %%%  where
      %%%  (args, results) = loop []
      %%%  f               = snd (last bnds)
      %%%  build args      = mkLets bnds (mkApps f args)
%%%
      %%%  -- (1) find the correct number of arguments
      %%%  loop :: [$v$] -> ([$v$], [($v$, $\su$)])
      %%%  loop args = case eval (build args, []) of
        %%%  ($\efun{x}{e}$, _) : _ -> loop (args `snoc` $\ehole{}$)
        %%%  results      -> (args, results)
  %%%  \end{mcode}
  %%%  \caption{A procedure for generating witnesses}
    %%%  \ES{should address case where output types of successive runs dont match}
  %%%  }
  %%%  \label{fig:search-algo}
%%%  \end{figure*}
%


% !TEX root = main.tex
