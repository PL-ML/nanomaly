\section{Related Work}
\label{sec:related-work}
In this section we connect our work to related efforts on type errors,
testing, and program exploration.

\subsection{Type Errors}
\label{sec:type-error}

\paragraph{Diagnosis and Repair}
\label{sec:diagnosis-repair}
It is well known that unification-based type inference procedures can
produce poor error messages, and in particular, can misidentify the
\emph{source} of the type error.
%
Thus, many groups have explored techniques to pinpoint the true source
of the error and recommend fixes.

Lerner \etal~\cite{lerner_searching_2007} attempt to localize the type
error and suggest a fix by replacing expressions (or removing them
entirely) with alternatives based on the surrounding program context.
%
Chen and Erwig~\cite{chen_counter-factual_2014} use a variational type
system represent the possibility that the inferred type of an expression
may be incorrect, because the expression may be the source of the
error. It then attempts to deduce the error source by searching for an
expression whose type can be changed such that type inference would
succeed. In contrast to Seminal, which searches for changes at the
value-level, by searching at the type level the search is complete due
the finite universe of types in a program.
%
Neubauer and Thiemann~\cite{neubauer_discriminative_2003} present a
decidable type system based on discriminative sum types, in which all
terms are typeable and type derivations contain all type errors in a
program. They then use the typing derivation to slice out the parts of
the expression related to each error.
%
Zhang and Myers~\cite{zhang_toward_2014} present an algorithm for
identifying the most likely culprit in a system of unsatisfiable
constraints (\eg type equalities), based on Bayesian reasoning.
%
Pavlinovic \etal~\cite{pavlinovic_finding_2014} translate the error
localization problem to a MaxSMT optimization problem, using
compiler-provided weights to rank the possible sources.
% \item \cite{chen_error-tolerant_2012}
% \item \cite{okeefe_type_1992}
% \item \cite{gomard_partial_1990}
% \item \cite{thatte_type_1988}

In contrast to these approaches, we do not attempt to localize or fix
the type error. Instead we try to explain it to the novice user using a
dynamic witness that demonstrates how their program is not just
ill-typed but truly wrong. In addition, allowing users to run their
program (even knowing that it is wrong) enables experimentation and the
use of debuggers to step through the program and investigate its
evolution.

\paragraph{Running Ill-Typed Programs}
\label{sec:running-ill-typed}
Vytiniotis \etal~\cite{vytiniotis_equality_2012} extend the Haskell
compiler GHC to support compiling ill-typed programs, but their intent
is rather different from ours. Their goal was to allow programmers to
incrementally test refactorings, which often cause type errors in
distant pieces of code. They replace any expression that fails to type
check with a \emph{runtime} error, but do not perform any type checking
at runtime.

\subsection{Testing}
\label{sec:testing}
\nanomaly is at its heart a test generator, and builds on a rich line of
work.
%
Our use of holes to represent unknown values is inspired by the work of
Runciman, Naylor, and Lindblad~\cite{runciman_smallcheck_2008,naylor_finding_2007,lindblad_property_2007},
%
who use lazy evaluation to drastically reduce the search space for
exhaustive test generation, by grouping together equivalent inputs by
the set of values they force. An exhaustive search is complete (up to
the depth bound), if a witness exists it will be found, but due to the
exponential blowup in the search space the depth bound can be quite
limited without advanced grouping and filtering techniques.
%
Our search is not exhaustive; instead we use random generation to fill
in holes on demand.
%
Random test generation~\cite{claessen_quickcheck:_2000,csallner_jcrasher:_2004,pacheco_feedback-directed_2007}
%
is by its nature incomplete, but is able to check larger inputs than
exhaustive testing as a result.

Instead of enumerating values, which may trigger the same path through
the program, one might enumerate paths. 
%
Dynamic-symbolic execution~\cite{godefroid_dart:_2005,cadar_klee:_2008,tillmann_pex_2008}
%
enables this by symbolic execution (to track which path a given input
vector triggers) with concrete execution (to ensure test failures are
not spurious). The system collects a path condition during execution,
which tracks symbolically what conditions must be met to trigger the
current path. Upon successfully completing a test run, it negates the
path condition and queries a solver for another set of inputs that
satisfy the negated path condition, \ie inputs that will not trigger the
same path. Thus, dynamic-symbolic execution can prune the search space
much faster than techniques based on enumerating values, but is limited
by the expressiveness of the underlying solver. Our operational
semantics is amenable to dynamic-symbolic execution, one would just need
to collect the path condition and replace our implementation of \gensym
by a call to the solver. We chose to use lazy, random generation instead
because it does not incur the overhead of an external solver, and
produces high coverage for our domain of novice programs.

% \begin{itemize}
% \item \cite{claessen_quickcheck:_2000}
% \item 
% \item \cite{godefroid_dart:_2005}
% \item \cite{cadar_klee:_2008}
% \item \cite{tillmann_pex_2008}
% \end{itemize}

\subsection{Program Exploration}
Perera \etal~\cite{perera_functional_2012} present a tracing semantics
for functional programs that tags values with their provenance, enabling
a form of backwards program slicing from a final value to the sequence
of reductions that produced it. Notably, they allow the user to supply a
\emph{partial value} -- a value containing holes -- and in turn present
a partial slice, containing only those steps that affected the specified
components of the partial value. This system is designed to answer
questions of the form ``Where did this value come from?'' and thus is
focused on backward exploration. In contrast, our trace visualization
supports forward \emph{and} backward exploration, though our backward
steps are more limited than Perera's. Specifically, we do not support
selecting a value and inserting the intermediate terms that preceded it
while ignoring unrelated computation steps; this would be interesting
future work.

\ES{todo: more on program exploration}

%%% Local Variables: 
%%% mode: latex
%%% TeX-master: "main"
%%% End: 
