\usepackage{listings}

\usepackage{color}
% uncomment next line to restore colors
\def\withcolor{}

\ifdefined\withcolor
	\definecolor{haskellblue}{rgb}{0.0, 0.0, 1.0}
	\definecolor{haskellstr}{rgb}{0.2, 0.2, 0.6}
	\definecolor{haskellred}{rgb}{1.0, 0.0, 0.0}
	\definecolor{gray_ulisses}{gray}{0.55}
	\definecolor{castanho_ulisses}{rgb}{0.71,0.33,0.14}
	\definecolor{preto_ulisses}{rgb}{0.41,0.20,0.04}
	\definecolor{green_ulisses}{rgb}{0.0,0.4,0.0}
\else
	\definecolor{haskellblue}{gray}{0.1}
	\definecolor{haskellstr}{gray}{0.1}
	\definecolor{haskellred}{gray}{0.1}
	\definecolor{gray_ulisses}{gray}{0.1}
	\definecolor{castanho_ulisses}{gray}{0.1}
	\definecolor{preto_ulisses}{gray}{0.1}
	\definecolor{green_ulisses}{gray}{0.1}
\fi


\def\codesize{\normalsize}

\lstdefinelanguage{HaskellUlisses}{
	basicstyle=\footnotesize\ttfamily,
	sensitive=true,
	morecomment=[l][\color{gray_ulisses}\ttfamily\footnotesize]{--},
	morecomment=[s][\color{gray_ulisses}\ttfamily\footnotesize]{\{-}{-\}},
	morestring=[b]",
	stringstyle=\color{haskellstr},
	showstringspaces=false,
	numberstyle=\codesize,
	numberblanklines=true,
	showspaces=false,
	breaklines=true,
	showtabs=false,
        escapeinside={(*}{*)},%
        % mathescape=true,
	emph=
	{[1]
		FilePath,IOError,abs,acos,acosh,and,any,appendFile,approxRational,asTypeOf,asin,
		asinh,atan,atan2,atanh,basicIORun,break,catch,ceiling,chr,compare,concat,concatMap,
		const,cos,cosh,curry,cycle,decodeFloat,denominator,digitToInt,div,divMod,drop,
		dropWhile,either,elem,encodeFloat,enumFrom,enumFromThen,enumFromThenTo,enumFromTo,
		error,even,exp,exponent,fail,filter,flip,floatDigits,floatRadix,floatRange,floor,
		fmap,foldl,foldl1,foldr,foldr1,fromDouble,fromEnum,fromInt,fromInteger,
		fromRational,fst,gcd,getChar,getContents,getLine,head,id,inRange,index,init,intToDigit,
		interact,ioError,isAlpha,isAlphaNum,isAscii,isControl,isDenormalized,isDigit,isHexDigit,
		isIEEE,isInfinite,isLower,isNaN,isNegativeZero,isOctDigit,isPrint,isSpace,isUpper,iterate,
		last,lcm,length,lex,lexDigits,lexLitChar,lines,log,logBase,lookup,mapM,mapM_,max,
		maxBound,maximum,maybe,min,minBound,minimum,mod,negate,not,notElem,numerator,odd,
		or,pi,primExitWith,print,product,properFraction,putChar,putStr,putStrLn,quot,
		quotRem,range,rangeSize,read,readDec,readFile,readFloat,readHex,readIO,readInt,readList,readLitChar,
		readLn,readOct,readParen,readSigned,reads,readsPrec,realToFrac,recip,rem,repeat,
		reverse,round,scaleFloat,scanl,scanl1,scanr,scanr1,seq,sequence,sequence_,show,showChar,showInt,
		showList,showLitChar,showParen,showSigned,showString,shows,showsPrec,significand,signum,sin,
		sinh,snd,span,splitAt,sqrt,subtract,succ,sum,tail,take,takeWhile,tan,tanh,threadToIOResult,toEnum,
		toInt,toInteger,toLower,toRational,toUpper,truncate,uncurry,undefined,unlines,until,unwords,unzip,
		unzip3,userError,words,writeFile,zip,zip3,zipWith,zipWith3,listArray,doParse,for,initTo,
                create,get,set,div,rescale,add,delete,insert,prop_focus_left_master,average,best,insert,union,split,size,fromList,copy,group,good,bad,foo,explode,singleton,difference,fromJust,sort,unfold,
                subst,unapply,apply,proxy,refinement,fresh,guard,constrain,oneOf,
                queryList,queryCtor,queryField,ctors,decodeCtor,whichOf,ctorArity,eval,
                mkCtor,gCtors,gEncode,gEncodeFields,gDecode,gDecodeFields,reproxyRep,empty,splitCtor,checkField,scanM,
                padAverage,focusUp,execute,checkSMT,inputTypes,outputType,toReft,app,
								genArgs, genWitness, close, witness, mkApps, mkLets, isStuck
                %, inTypes, inputTypes, outputType, execute, smtFindModel, smtRefuteModel
	},
	emphstyle={[1]\color{haskellblue}},
	emph=
	{[2]
		OkMap,OkRBT,OkStackSet,TTrue,Map,Bool,Char,Double,Either,Float,IO,Integer,Int,Maybe,Ordering,Rational,Ratio,ReadS,ShowS,String,Word8,Nat,Pos,Rng,Score,
                Ptr,ForeignPtr,CSize,InPacket,Tree,Prop,TreeEq,TreeLt,Vec,
                NullTerm,IncrList,DecrList,UniqList,BST,MinHeap,MaxHeap,
                PtrN,ByteStringN,ByteStringEq,VO,ByteStringsEq,ByteStringNE,OrdList,Var,RType,Constrain,Gen,Var,Proxy,SMT,Targetable,RefType,Refinement,Ctor,C1,Rep,Rec0,U1,
                GCtors,GDecode,GDecodeFields,GEncode,GEncodeFields,OrdMap,MinusKey,
                len,isBH,isBal,bh,isRB,keys,Sorted,RBT,Col,isBlack,OrdRBT,Set,sz,
                StackSet,NoDuplicates,Data,RBTree,XMonad,Generic,true
	},
	emphstyle={[2]\color{castanho_ulisses}},
	emph=
	{[3]
		case,class,data,deriving,do,else,if,return,def,import,in,infixl,infixr,instance,let,tmapM,for2M,forM,zipWithM,otherwise,
		module,measure,pred,predicate,of,primitive,then,type,where,lazy,throw,when,
                rec,function,fun,match,with
	},
	emphstyle={[3]\color{preto_ulisses}\textbf},
	emph=
	{[4]
		quot,rem,div,mod,elem,notElem,seq
	},
	emphstyle={[4]\color{castanho_ulisses}\textbf},
	emph=
	{[5]
		PS,Tip,Node,Black,Red,EQ,False,GT,Just,LT,Left,Nothing,Right,True,Show,Eq,Ord,Num,C,N,Leaf,Bin,CounterExample
	},
	emphstyle={[5]\color{green_ulisses}},
	emph=
	{[6]
		patError, irrefutPatError, nonExhaustiveGuardsError, recSelError, errorOut,
		noMethodBinding
	},
	emphstyle={[6]\color{haskellred}}
}

%%%ORIG
%%%\lstnewenvironment{code}
%%%{\textbf{Haskell Code} \hspace{1cm} \hrulefill \lstset{language=HaskellUlisses}}
%%%{\hrule\smallskip}

%V1
%\lstnewenvironment{code}
%{\smallskip \lstset{language=HaskellUlisses}}
%{\smallskip}

\lstdefinelanguage{HaskellUlissesMath}[]{HaskellUlisses}{mathescape=true}

\lstnewenvironment{code}
{\lstset{language=HaskellUlisses}}
{}

\lstnewenvironment{mcode}
{\lstset{language=HaskellUlissesMath}}
{}

\lstnewenvironment{fcode}
{\lstset{language=HaskellUlisses, frame=L}}
{}

\lstMakeShortInline[language=HaskellUlisses]@
\lstMakeShortInline[language=HaskellUlissesMath]|
