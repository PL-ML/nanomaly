\section{Type-Error Witnesses}
\label{sec:searching-witness}

% Our goal is to find concrete values that demonstrate how a
% program ``goes wrong''.

% \paragraph{Problem: Which inputs are bad?}
% %
% One approach is to randomly generate input values and use
% them to execute the program until we find one that causes
% the program to go wrong. However, to see why this approach
% is naive, consider the following example:
% %
% \begin{lstlisting}
%   let f x =
%     let y = 1 + x in
%       1. +. y
% \end{lstlisting}
% %
% What \emph{types} of inputs should we test \texttt{f} with?
% Values of type \texttt{int} and \texttt{float} are fair game,
% but values of type say, \texttt{string} or \texttt{int list}
% will cause the program to go wrong in an \emph{irrelevant}
% manner.

% \paragraph{Solution:} \RJ{STOP}


% we cannot provide \emph{completely arbitrary} inputs to
% \texttt{f}. Instead, we call \texttt{f} with a \emph{hole}, written
% \ehole{}, which is a placeholder for a value whose type we have not
% yet determined. As we execute the program, we instantiate holes with
% concrete values as demanded by the primitive operations in the
% program. For example, the hole we pass to f will be instantiated to an
% int when we reach the \lstinline{1 + x} term. Thus, y will be an int as
% well, and the program will get stuck at \lstinline{1. +. y}. \ES{this
%   reads more like overview text..}
%


% \begin{itemize}
% \item how do we run ill-typed programs?
% \item for a lang like ocaml, dynamic semantics are independent of static
%   semantics, just lambda calculus. so no problem to run ill-typed
%   program
% \item but what about functions? what type of arguments should we pass? consider
%
% \begin{lstlisting}
% let f x =
%   let y = 1 + x in
%     1. +. y
% \end{lstlisting}
%
% does \texttt{f} take an int, float, string? int and float are both
% somewhat plausible, but string or anything else is ``clearly'' bogus. so
% we cannot provide \emph{completely arbitrary} inputs to
% \texttt{f}. Instead, we call \texttt{f} with a \emph{hole}, written
% \ehole{}, which is a placeholder for a value whose type we have not
% yet determined. As we execute the program, we instantiate holes with
% concrete values as demanded by the primitive operations in the
% program. For example, the hole we pass to f will be instantiated to an
% int when we reach the \lstinline{1 + x} term. Thus, y will be an int as
% well, and the program will get stuck at \lstinline{1. +. y}. \ES{this
%   reads more like overview text..}
%
% % \item values are tagged with their types, just like ``untyped'' langs
% % \item special ``hole'' value whose type is not yet known, used for function args
% % \item on-the-fly unification to determine ``correct'' type for holes
% \end{itemize}


Next, we formalize the notion of type error witnesses as follows.
%
First, we define a core calculus within which we will work~\S~\ref{sec:syntax}.
%
Second, we develop a (non-deterministic) operational semantics
for ill-typed programs that precisely defines the notion
of a \emph{witness}~\S~\ref{sec:semantics},
%
Third, we formalize and prove a notion of \emph{generality} for
witnesses, which states, intuitively that if we find a
single witness then for \emph{every possible} type
assignment, there exist inputs that are guaranteed to make
the program ``go wrong''~\S~\ref{sec:soundness}, and finally,
%
Fourth, we refine the operational semantics into a
\emph{search procedure} that returns concrete (general)
witnesses for ill-typed programs~\S~\ref{sec:search-algorithm}.

\subsection{Syntax}
\label{sec:syntax}
\begin{figure}
% \hrule width 0.48\textwidth \vspace{0.05in}
$$
\begin{array}{rrcl}
% \emphbf{Configurations} \quad
%   & c & ::=    & \triple{e}{\vsu}{\tsu} \spmid \triple{\stuck}{\vsu}{\tsu} \\[0.05in]

\emphbf{Expressions}
  & \estuck & ::= & e \spmid \stuck \\
  & e & ::=    & v \spmid x \spmid \eapp{e}{e} \spmid \eplus{e}{e}\\
  &   & \spmid & \eif{e}{e}{e} \\
  % &   & \spmid & \elet{x}{e}{e} \\
  &   & \spmid & \epair{e}{e} \spmid \epcase{e}{x}{x}{e} \\
  &   & \spmid & \enode{e}{e}{e} \spmid \eleaf \\
  &   & \spmid & \ecase{e}{e}{x}{x}{x}{e} \\[0.1in]

\emphbf{Values}
  & v  & ::= & n \spmid b \spmid \efun{x}{e} \spmid \vhole{\thole} \spmid tr \\
  & tr & ::= & \vnode{t}{v}{v}{v} \spmid \vleaf{t} \\[0.05in]

\emphbf{Integers}
  & n & ::= &  0,1,-1,\ldots \\[0.05in]

\emphbf{Booleans}
  & b & ::= &  \etrue \spmid \efalse \\[0.05in]

\emphbf{Types}
  & t & ::=     & \tbool \spmid \tint \spmid \tfun \\
  &   &  \spmid & \tprod{t}{t} \spmid \ttree{t} \spmid \thole \\[0.05in]

\emphbf{Substitutions}
  & \vsu & ::= & \emptysu \spmid \extendsu{\vsu}{\vhole{\thole}}{v} \\
  & \tsu & ::= & \emptysu \spmid \extendsu{\tsu}{\thole}{t} \\[0.1in]
% \end{array}
% $$
% % \hrule width 0.48\textwidth
% $$
% \begin{array}{rrcl}
\emphbf{Contexts}
  & C
  & ::=
  &   	 \bullet
  \spmid \eapp{C}{e}
  \spmid \eapp{v}{C} \\
  & & \spmid & \eplus{C}{e} \spmid \eplus{v}{C} \\
  & & \spmid & \eif{C}{e}{e} \\
  % & & \spmid & \elet{x}{C}{e} \\
  & & \spmid & \epair{C}{e} \spmid \epair{v}{C} \\
  & & \spmid & \epcase{C}{x}{x}{e} \\
  & & \spmid & \enode{C}{e}{e} \\
  & & \spmid & \enode{v}{C}{e} \\
  & & \spmid & \enode{v}{v}{C} \\
  & & \spmid & \ecase{C}{e}{x}{x}{x}{e} \\[0.05in]

\emphbf{Type Contexts}
  & T &::=& \bullet \spmid \ttree{T} \spmid \tprod{T}{t} \spmid \tprod{t}{T} \\[0.05in]
\end{array}
$$

% \judgementHead{Reduction}{\eval{e}{e}}

% $$
% \begin{array}{rcl}
% \eval{C[e]&}{&C[e']} \qquad \text{if}\ \eval{e}{e'} \\
% 	\eval{\eapp{c}{v}&}{& \ceval{c}{v}}\\
% \eval{\eapp{(\efun{x}{\tau_x}{e})}{e_x}&}{&e\sub{x}{e_x}}\\
% 	\eval{\elet{x}{e_x}{e}&}{&e\sub{x}{e_x}} \\
% 	\eval{\ecase{D_j\ \overline{e}}{D_i}{\overline{y_i}}{e_i}{x}&}
% 	{&e_j\sub{x}{D_j\ \overline{e}}\sub{\overline{y_j}}{\overline{e}}} \\
% \end{array}
% $$

\caption{Syntax of \lang}
\label{fig:syntax}
\end{figure}

%
Figure~\ref{fig:syntax} describes the syntax of \lang, a simple lambda
calculus with integers and booleans.
%
As we are specifically interested in programs that \emph{do} go wrong,
we include an explicit \stuck\ state in our syntax.

\paragraph{Holes}
\label{sec:holes}
%
Recall that a key challenge in our setting is to find witnesses
that are meaningful and do not arise from choosing values from
irrelevant types.
%
We solve this problem be equipping our term language with a notion
of a \emph{hole}, written \ehole{}, which represent \emph{unconstrained}
values that may replaced with values from \emph{any} type.
%
Dually, holes may also appear in types, where they may be thought
of as type variables that we will not generalize over.
%

\paragraph{Substitutions}
%
Our semantics ensure the generality of witnesses by incrementally
\emph{refining} holes filling in just as much information as is
needed locally to make progress (inspired by the manner in
which SmallCheck uses lazy evaluation~\cite{smallcheck}).
%
We track how the holes are incrementally filled in, by using
(value) \emph{substitutions} that map holes (denoted by their index)
to values which may, in turn, contain holes.
%
The substitutions let us ensure that we instantiate each hole
with the same (partially defined) value, regardless of the multiple
contexts in which the hole appears, and hence, ensures we can
report a concrete (and general) witness for any (dynamically)
discovered type errors.

\subsection{Semantics}
\label{sec:semantics}
\begin{figure*}
\judgementHead{Evaluation}{\step{e}{\su}{e}{\su}}
\begin{gather*}
\inference[\recontext]
  {\step{e}{\su}{e_1}{\su_1}}
  {\step{C[e]}{\su}{C[e_1]}{\su_1}}
\qquad
\inference[\restuck]
  {}
  {\step{C[\stuck]}{\su}{\stuck}{\su}}
\\ \\
\inference[\replusgood]
  {\pair{n_1}{\su_2} = \force{v_1}{\tint} \\
   \pair{n_2}{\su_3} = \force{v_2}{\tint} \\ 
   n = \eplus{n_1}{n_2}}
  {\step{\eplus{v_1}{v_2}}{\su_1}{n}{\su_1;\su_2;\su_3}}
\qquad
\inference[\replusbadone]
  {\pair{\stuck}{\su_2} = \force{v_1}{\tint}}
  {\step{\eplus{v_1}{v_2}}{\su_1}{\stuck}{\su_1;\su_2}}
\\ \\
\inference[\replusbadtwo]
  {\pair{\stuck}{\su_2} = \force{v_2}{\tint}}
  {\step{\eplus{v_1}{v_2}}{\su_1}{\stuck}{\su_1;\su_2}}
\qquad
\inference[\reifgoodone]
  {\pair{\etrue}{\su_2} = \force{v}{\tbool}}
  {\step{\eif{v}{e_1}{e_2}}{\su_1}{e_1}{\su_1;\su_2}}
\\ \\
\inference[\reifgoodtwo]
  {\pair{\efalse}{\su_2} = \force{v}{\tbool}}
  {\step{\eif{v}{e_1}{e_2}}{\su_1}{e_2}{\su_1;\su_2}}
\qquad
\inference[\reifbad]
  {\pair{\stuck}{\su_2} = \force{v}{\tbool}}
  {\step{\eif{v}{e_1}{e_2}}{\su_1}{\stuck}{\su_1;\su_2}}
\\ \\
\inference[\reappgood]
  {\pair{\efun{x}{e}}{\su_2} = \force{v_1}{\tfun{\thole{}}{\thole{}}}}
  {\step{\eapp{v_1}{v_2}}{\su_1}{e\sub{x}{v_2}}{\su_1;\su_2}}
\qquad
\inference[\reappbad]
  {\pair{\stuck}{\su_2} = \force{v_1}{\tfun{\thole{}}{\thole{}}}}
  {\step{\eapp{v_1}{v_2}}{\su_1}{\stuck}{\su_1;\su_2}}
\\ \\
\inference[\relet]
  {}
  {\step{\elet{x}{v}{e}}{\su}{e\sub{x}{v}}{\su}}
\end{gather*}
\\ % [0.05in]
\relDescription{\forcesym and \gensym}
\begin{gather*}
\begin{array}{lcl}
\force{\ehole{i}}{t} & \defeq & \elet{v}{\gen{t}}{\pair{v}{\ehole{i} \mapsto v}} \\
\force{v}{\ehole{}}  & \defeq & \pair{v}{\emptysu} \\
\force{n}{\tint}    & \defeq & \pair{n}{\emptysu} \\
\force{v}{\tint}    & \defeq & \pair{\stuck}{\emptysu} \\
\force{b}{\tbool}   & \defeq & \pair{b}{\emptysu} \\
\force{v}{\tbool}   & \defeq & \pair{\stuck}{\emptysu} \\
\force{\efun{x}{e}}{\tfun{\thole{}}{\thole{}}} & \defeq & \pair{\efun{x}{e}}{\emptysu} \\
\force{v}{\tfun{\thole{}}{\thole{}}} & \defeq & \pair{\stuck}{\emptysu} \\
\end{array}
\qquad
\begin{array}{lcll}
\gen{\tint}   & \defeq & n & \\
\gen{\tbool}  & \defeq & b & \\
\gen{\tfun{t_1}{t_2}} & \defeq & \efun{x}{\ehole{i}}, & \quad \text{$i$ is fresh} \\
\gen{\thole{}} & \defeq & \ehole{i}, & \quad \text{$i$ is fresh} \\
\end{array}
\end{gather*}
\caption{Evaluation relation}
\label{fig:operational}
\end{figure*}

%

The evaluation relation is parameterized by a pair of functions:
called \emph{narrow} (\forcesym) and \emph{generate} (\gensym),
that ``dynamically'' perform type-checking and hole-filling
respectively.

\paragraph{Narrowing Types} The procedure $\force{e}{t}{\vsu}$ takes as
input an term \texttt{e} and refines it to have the type \texttt{t} by
yielding a pair of either the same expression and a value substitution
the refines the holes in \texttt{e} to ensure it has type \texttt{t}, or
yields the stuck term if no such refinement is possible. In the case
where \texttt{e} is a hole, it first checks in the given $\vsu$ to see
if the hole has already been instantiated and, if so, returns the
existing instantiation:
%
$$
\begin{array}{lcl}
% \multicolumn{3}{l}{\forcesym \ :: \  (e, t) \rightarrow \pair{e}{\vsu}} \\
% \forcesym            & ::     & (e, t) \rightarrow \pair{e}{\vsu} \\
\force{\ehole{i}}{t}{\vsu} & \defeq & \hspace{-1ex}
\begin{cases}
  \pair{\vsu(i)}{\emptysu}        & \mbox{if}\ i \in \vsu \\
  \pair{v}{\ehole{i} \mapsto v}   & v = \gen{t}\\
\end{cases} \\
 % \begin{array}{l}
    % \mathtt{if}\ i \in \vsu\ \mathtt{then}\ \pair{\vsu(i)}{\emptysu}\ \mathtt{else} \\
         % \elet{v}{\gen{t}}{\pair{v}{\ehole{i} \mapsto v}}
  % \end{array} \\
\force{v}{\ehole{}}{\vsu}  & \defeq & \pair{v}{\emptysu} \\
\force{n}{\tint}{\vsu}     & \defeq & \pair{n}{\emptysu} \\
\force{v}{\tint}{\vsu}     & \defeq & \pair{\stuck}{\emptysu} \\
\force{b}{\tbool}{\vsu}    & \defeq & \pair{b}{\emptysu} \\
\force{v}{\tbool}{\vsu}    & \defeq & \pair{\stuck}{\emptysu} \\
\force{\efun{x}{e}}{\tfun{\thole{}}{\thole{}}}{\vsu} & \defeq & \pair{\efun{x}{e}}{\emptysu} \\
\force{v}{\tfun{\thole{}}{\thole{}}}{\vsu} & \defeq & \pair{\stuck}{\emptysu} \\
\end{array}
$$

\paragraph{Generating Values} The (non-deterministic) procedure $\gen{t}$
takes as input a type \texttt{t} and returns a value of that type. For base
types, the procedure returns an arbitrary value. For functions, it returns
a lambda with a \emph{new} hole denoting the return value, and unconstrained
types (denoted by $\thole{}$) yield fresh, unconstrained values
(denoted by $\ehole{i}$).

$$
\begin{array}{lcll}
% \gensym       & ::      & t \rightarrow v \\
\gen{\tint}   & \defeq  & n, & \quad \text{non-deterministic} \\
\gen{\tbool}  & \defeq  & b, & \quad \text{non-deterministic} \\
\gen{\tfun{t_1}{t_2}}   & \defeq & \efun{x}{\ehole{i}}, & \quad \text{$i$ is fresh} \\
\gen{\thole{}} & \defeq & \ehole{i}, & \quad \text{$i$ is fresh} \\
\end{array}
$$

\paragraph{Steps and Traces}
%
Figure~\ref{fig:operational} describes the small-step contextual
reduction semantics for \lang.
%
A \emph{closed expression} is a pair $\pair{e}{\vsu}$ of an expression $e$
and a substitution $\vsu$.
%
We write $\step{e}{\vsu}{e'}{\vsu'}$ if the closed expression $\pair{e}{\vsu}$
transitions in a \emph{small step} to $\pair{e'}{\vsu'}$.
%
A (finite) \emph{trace} $\trace$ is a sequence of closed expressions
$\pair{e_0}{\vsu_0}, \ldots, \pair{e_n}{\vsu_n}$ such that
$\forall 0 \leq i < n$, we have \step{e_i}{\vsu_i}{e_{i+1}}{\vsu_{i+1}}.
%
We write \steptr{\trace}{e}{\vsu}{e'}{\vsu'} if $\trace$ is a trace of the form
$\pair{e}{\vsu},\ldots,\pair{e'}{\vsu'}$.
%
We write \steps{e}{\vsu}{e'}{\vsu'} if \steps{e}{\vsu}{e'}{\vsu'}
for some trace $\trace$.

\paragraph{Primitive Reductions}
Primitive reduction steps -- addition, if-elimination, and
function application -- use \forcesym to ensure that values have the
appropriate type (and that holes are instantiated) before continuing the
computation. Importantly, beta-reduction \emph{does not} type-check its
argument, it only ensures that the value being applied is a function.

%% \begin{thm}
%% \label{thm:all-reduce}
%%   Every closed expression $e$ reduces to a value $v$ (which may be \stuck).
%% \ES{do we really need to state this, or is it obvious?}
%% \end{thm}

% \begin{proof}%[Proof of \autoref{thm:all-reduce}]
%   Simple induction on the evaluation relation.
% \end{proof}
%
\subsection{Generality}\label{sec:soundness}

A key technical challenge in generating witnesses is
that we have no (static) type information to rely upon. Thus, we
must avoid the trap of generating \emph{spurious} witnesses that
arise from picking irrelevant values, when instead there exist
perfectly good values of a \emph{different} type, under which
the program would not go wrong.

\paragraph{Witness Generality}
We now show that our evaluation relation instantiates holes in a
\emph{general} manner. That is, given a function $f$, if we have
$\steps{\eapp{f}{\ehole{}}}{\emptysu}{\stuck}{\su}$, then for every
possible input type $t$, we can find a value $\hastype{v}{t}$
such that $\eapp{f}{v}$ goes wrong.

\begin{thm}{\textbf{[Witness Generality]}}
\label{thm:soundness}
  For any function $f$, if \steptr{\trace}{\eapp{f}{\ehole{}}}{\emptysu}{\stuck}{\su},
  then for every type $t$ there exists $\hastype{v}{t}$ such that
  $\steps{\eapp{f}{v}}{\emptysu}{\stuck}{\su}$.
\end{thm}

We need to develop some machinery in order to prove this theorem.
First, we show how our evaluation rules encode a dynamic form of
type inference, and next, we show that the types inferred via
evaluation are indeed maximally general.

\paragraph{The Type of an Expression} The (dynamic) type of an
expression $e$ is defined as a function $\typeof{e}$ defined as:
  \[
  \begin{array}{lcll}
    \typeof{n}   & \defeq & \tint & \\
    \typeof{b}   & \defeq & \tbool & \\
    \typeof{\efun{x}{e}} & \defeq & \tfun{\thole{i}}{\typeof{e}}, & \quad \text{$i$ is fresh} \\
    \typeof{e} & \defeq & \thole{i}, & \quad \text{$i$ is fresh} \\
  \end{array}
  \]

\paragraph{Dynamic Type Inference}
We can think of the evaluation of \eapp{f}{\ehole{}} as (dynamically)
inferring a \emph{partial type} --- a type that may contain holes ---
for $f$.
%
We can extract this type from evaluation traces.
%
Formally, we say that if \steptr{\trace}{\eapp{f}{\ehole{}}}{\emptysu}{e}{\su},
then the \emph{partial type} of $f$ upto $\trace$, written \ptype{\trace}{f},
is \tfun{\typeof{\subst{\su}{\ehole{}}}}{\typeof{e}}.
%
%We will omit the subscript when we wish to refer to the final partial
%type, \ie\ at the step where the expression has been reduced to a value
%(or stuck.)

\paragraph{Type Compatibility}
Two types are compatible if one of them can be obtained by
suitably filling the holes of the other (\ie by suitably ``instantiating''
the type variables appearing in the other). Formally, a type $s$ is
\emph{compatible} with a type $t$, written \tcompat{s}{t}, if
$\exists \tsu.\ t = \subst{\tsu}{s} \lor s = \subst{\tsu}{t}$.

\paragraph{Preservation}
Given these two definitions, we show that each evaluation step
refines the partial type of $f$, \ie preserves type compatibility.
%
\begin{lem}
\label{lem:refine-partial}
If $\trace \defeq \pair{\eapp{f}{\ehole{}}}{\emptysu},\ldots$ and
$\trace' \defeq \trace, \pair{e'}{\vsu'}$ (\ie $\trace'$ is single
step extension of $\trace$)
%
%The partial type of $f$ upto $\trace$ is compatible
%with the partial type upto $\trace'$, \ie\
%
then \tcompat{\ptype{\trace}{f}}{\ptype{\trace'}{f}}.
\end{lem}
\begin{proof}
  By induction on the length of $\trace$ and case analysis on the evaluation rules.
  %
  Note that all rules preserve partial types with the exception of when
  \forcesym\ is called on a hole, in which case we instantiate the hole with
  a concrete value.
  %
  But \hastype{\ehole{}}{\thole{}}, which is compatible with any type.
\end{proof}

\paragraph{Narrowing}
%
Furthermore, only a call to \forcesym can change the partial type of $f$.
%
\begin{lem}
\label{lem:force-inst}
If
$\trace \defeq \pair{\eapp{f}{\ehole{}}}{\emptysu}, \ldots,\pair{e}{\vsu}$
and
$\trace' \defeq \trace, \step{e}{\vsu}{e'}{\vsu'}$
and
$\ptype{\trace}{f} \neq \ptype{\trace'}{f}$,
then the final step $\step{e}{\vsu}{e'}{\vsu'}$ invokes \forcesym.
\end{lem}

\begin{proof}
  By case analysis on the evaluation rules.
  %
  If $\ptype{\trace}{f} \neq \ptype{\trace}{f}$ then one of the holes in $f$'s
  argument must have been instantiated with a concrete value at the last step.
  %
  An examination of the rules shows that only place this happens is
  in the first case of \forcesym.
\end{proof}

\paragraph{Incompatible Values Are Wrong}
%
\emph{Any} value that is \emph{incompatible} with
the partial type upto trace $\trace$ will cause $f$ to get stuck
in \emph{at most} $k+1$ steps where $k$ is the length of $\trace$.
%
\begin{lem}
\label{lem:k-stuck}
  If \steptr{\trace}{\eapp{f}{\ehole{}}}{\emptysu}{e}{\vsu},
     \tincompat{t}{\ptype{\trace}{f}}, and
     \hastype{v}{t},
  then
     \steps{\eapp{f}{v}}{\emptysu}{\stuck}{\su}.
\end{lem}
\ES{self-todo: take a close look at the $\trace$'s vs k's}
\begin{proof}
  By induction on $k$, the length of $\trace$.
  Let $\tfun{s_{\trace}}{\_} = \ptype{\trace}{f}$.
  Suppose \hastype{v}{\tincompat{t}{s_{\trace'}}}, we
  will show that \steps{\eapp{f}{v}}{\emptysu}{\stuck}{\su'}
  in at most $k+1$ steps:
%
  %\begin{description}
  % \item
  [Case \tincompat{t}{s_{\trace}}:]
    The inductive hypothesis applies.
  %\item
  [Case \tcompat{t}{s_{\trace}} but \tincompat{t}{s_{\trace'}}:]
    By Lemma~\ref{lem:force-inst} we must have called \forcesym at step
    $k+1$.
    %
    A case analysis of the applicable rules shows that \forcesym cannot
    have succeeded.
  %\end{description}
\end{proof}

\paragraph{Proof of Theorem~\ref{thm:soundness}}
%
Suppose $\trace$ witnesses that $f$ gets stuck,
and let $t$ be the partial type of $f$ upto $\trace$.
We show that \emph{all} types $s$ have stuck-inducing
values by splitting cases on whether the type is
compatible with the partial type upto $\trace$.
%
% \begin{description}
  %\item
  [Case \tcompat{s}{t}:]
   The value $\hastype{v}{t}$ suffices to demonstrate that $\eapp{f}{v}$
   gets stuck.
  %\item
  [Case \tincompat{s}{t}:]
   By Lemma~\ref{lem:k-stuck}, every \hastype{v}{t} demonstrates that
   $\eapp{f}{v}$ gets stuck.
%\end{description}


\subsection{Search Algorithm}
\label{sec:search-algorithm}
%
So far, we have seen how a trace leading to a stuck configuration yields
a general witness demonstrating that the program is ill-typed (\ie\ goes
wrong for all inputs.)
In particular, we have shown how to non-deterministically find a witnesses
for a function of a \emph{single} argument.

In order to convert the semantics into a \emph{procedure} for finding
witnesses, we must address two challenges.
%
First, we must resolve the non-determinism introduced by \gensym.
%
Second, in the presence of higher-order functions and currying,
we must determine how many concrete values to generate to make
execution go wrong (as we cannot rely upon static typing to
provide this information.)

The witness generation procedure @genWitness@ is formalized in
Figure~\ref{fig:algo-gen-witness}.
%
Next, we describe what it takes as input and output, and how it
addresses the above challenges to search the space of possible
executions for general type error witnesses.

\paragraph{Inputs and Outputs}
%
Obviously, the problem of generating inputs is undecidable in general.
%
The witness generation procedure takes two inputs.
%
First, a search bound @n@ which is used to define the \emph{number} of
traces to explore. (We assume, without loss of generality, that all
traces are finite.)
%
Second, binder-expression tuples @[(f1,e1),...,(fn,en)]@
denoting the list of \emph{top-level definitions}
%
\begin{code}
  let f1 = e1 in
  let f2 = e2 in
  ...
  let fn = en in
\end{code}
%
The witness generation procedure returns as output a list of (general)
witness expressions $e$ each of which is of the form @fn v1 ... vn@.
%
The \emph{empty} list is returned when no witness can be found after
exploring $n$ traces.
%
In the sequel, we refer to @fn@ as the \emph{target} of witness generation.

\paragraph{Modeling Semantics}
%: \texttt{eval} and \texttt{isStuck}}
%
We resolve the non-determinism in the operational semantics
(\S~\ref{sec:semantics}) via the procedure:
%
\begin{mcode}
  eval :: ($e$, $\su$) -> [($v$, $\su$)]
\end{mcode}
%
Due to the non-determinism introduced by \gensym, a call \hbox{|eval($e$, $\vsu$)|}
returns the \emph{list} possible results of the form $\pair{e'}{\vsu'}$ such
that $\steps{e}{\vsu}{e'}{\vsu'}$.
%
We determine whether a given value is stuck by using the predicate
%
\begin{mcode}
  isStuck :: $v$ -> Bool
\end{mcode}

\paragraph{Generating Target Arguments}
%
We address the issue of currying by defining a procedure @genArgs@ such that \\
\hbox{@genArgs [(f1,e1),...,(fn,en)]@} yields a list of holes @[v1,...,vk]@
such that \hbox{@fn v1 ... vk@} \emph{does not} evaluate to a lambda.
%
This is achieved via a simple @loop@ that keeps adding holes until
@eval@uating the closed term yields a non-lambda value.
%
The helper functions @witness@ and @close@ are used to respectively
apply the parameters to the target and close the result under the top-level
let-binders, prior to invoking @eval@.
%
That is, @mkApps@ creates a nested sequence of applications in
the usual left-associative style, and @mkLets@ takes a list of
binders and a body expression, and creates a sequence of nested
let-binders that close the body expression.

\begin{figure}[t]
\centering
\begin{mcode}
genArgs :: [($x$, $e$)] -> [$v$]
genArgs bs = loop []
  where
  loop vs  = case eval (close bs vs) of
               $\pair{\efun{x}{e}}{\_}$:_ -> loop ($\ehole{}$:vs)
               _        -> vs

close :: [($x$, $e$)] -> [$v$] -> ($e$, $\vsu$)
close bs vs = (e, $\emptysu$)
  where
    e       = mkLets bs (witness bs vs)

witness :: [($x$, $e$)] -> [$v$] -> $e$
witness bs vs = mkApps target vs
  where
    target    = snd (last bs)
\end{mcode}
\caption{Find number of arguments for target function}
\label{fig:algo-gen-args}
\end{figure}


\paragraph{Generating Witnesses}
% Procedure \texttt{genWitness}}
%
Finally, Figure~\ref{fig:algo-gen-witness} summarizes the overall
implementation of our search for witnesses, in the procedure @genWitness@,
which takes as input a bound @n@ and a list of top level binders, and
returns a list of witness expressions $e$ that demonstrate how the input
program gets stuck.
%
The search proceeds as follows.
%
@(1)@ we invoke @genArgs@ to determine the number of parameters required
by the (curried) target function.
%
@(2)@ we obtain a closed expression $e$ by closing the target with the
computed holes.
%
@(3)@ we take the first @n@ traces returned by @eval@ on the closed $e$
and
@(4)@ we extract the substitutions corresponding to the $\stuck$ traces,
and use them to return the list of witnesses.
%
We obtain the following as a corrollary of Theorem~\ref{thm:soundness}:

\begin{thm}{\textbf{[Witness Generation]}}
\label{thm:generation}
  If @genWitness k@ $[(f_1, e_1), \ldots, (f_n, e_n)]$ is non-empty,
  then for every type $t$ there exists $\hastype{v}{t}$ such that
  $\steps{\eapp{f_n}{v}}{\emptysu}{\stuck}{\su}$.
\end{thm}

\begin{figure}[t]
\centering
\begin{mcode}
genWitness :: Nat -> [($x$, $e$)] -> [$v$]
genWitness n bs
       = [witness bs ($\vsu$ vs) | $\vsu$ <- $\vsu$s]
  where
   vs  = genArgs defs []      -- (1)
   $e$    = close defs vs        -- (2)
   res = take n (eval e)      -- (3)
   $\vsu$s   = [$\vsu$ | $\pair{\stuck}{\vsu}$ <- res]  -- (4)
\end{mcode}
\caption{Generating Witnesses}
\label{fig:algo-gen-witness}
\end{figure}

%% \begin{figure}[t]
  %% \centering
  %% \begin{mcode}
  %% -- transitive small-step reduction, returning a list of results
  %% eval :: ($e$, $\su$) -> [($v$, $\su$)]
%%
  %% -- | is a value stuck?
  %% isStuck :: $v$ -> Bool
%%
  %% mkApps  :: $e$ -> [$e$] -> $e$
  %% mkLets  :: [($x$, $e$)] -> $e$ -> $e$
  %% \end{mcode}
  %% \caption{Expression API}
  %% \label{fig:expression-api}
%% \end{figure}
%% %
%% We also define a few helper functions for manipulating expressions:
%% \begin{itemize}
%% \item @subst@ applies a substitution of holes to a value,
%% \item @mkApps@ creates a nested sequence of applications in the usual
  %% left-associative style,
%% \item @mkLets@ takes a list of binders and a body expression, and
  %% creates a sequence of nested let-binders, and
%% \item @isStuck@ tests whether a value is the \stuck term.
%% \end{itemize}

%%%  \begin{figure*}[t]
  %%%  \centering
  %%%  \begin{mcode}
  %%%  check :: [($x$, $e$)] -> Result
  %%%  check bnds =
    %%%  -- (2) search for a witness
    %%%  case find (isStuck . fst) results of
      %%%  Nothing      -> Safe
      %%%  Just (_, su) -> Unsafe (mkApps f (subst su args))
    %%%  where
      %%%  (args, results) = loop []
      %%%  f               = snd (last bnds)
      %%%  build args      = mkLets bnds (mkApps f args)
%%%
      %%%  -- (1) find the correct number of arguments
      %%%  loop :: [$v$] -> ([$v$], [($v$, $\su$)])
      %%%  loop args = case eval (build args, []) of
        %%%  ($\efun{x}{e}$, _) : _ -> loop (args `snoc` $\ehole{}$)
        %%%  results      -> (args, results)
  %%%  \end{mcode}
  %%%  \caption{A procedure for generating witnesses}
    %%%  \ES{should address case where output types of successive runs dont match}
  %%%  }
  %%%  \label{fig:search-algo}
%%%  \end{figure*}
%


% !TEX root = main.tex
