\section{Searching for Type-Error Witnesses}
\label{sec:searching-witness}
% \begin{itemize}
% \item how do we run ill-typed programs?
% \item for a lang like ocaml, dynamic semantics are independent of static
%   semantics, just lambda calculus. so no problem to run ill-typed
%   program
% \item but what about functions? what type of arguments should we pass? consider
%
% \begin{lstlisting}
% let f x =
%   let y = 1 + x in
%     1. +. y
% \end{lstlisting}
%
% does \texttt{f} take an int, float, string? int and float are both
% somewhat plausible, but string or anything else is ``clearly'' bogus. so
% we cannot provide \emph{completely arbitrary} inputs to
% \texttt{f}. Instead, we call \texttt{f} with a \emph{hole}, written
% \ehole{}, which is a placeholder for a value whose type we have not
% yet determined. As we execute the program, we instantiate holes with
% concrete values as demanded by the primitive operations in the
% program. For example, the hole we pass to f will be instantiated to an
% int when we reach the \lstinline{1 + x} term. Thus, y will be an int as
% well, and the program will get stuck at \lstinline{1. +. y}. \ES{this
%   reads more like overview text..}
%
% % \item values are tagged with their types, just like ``untyped'' langs
% % \item special ``hole'' value whose type is not yet known, used for function args
% % \item on-the-fly unification to determine ``correct'' type for holes
% \end{itemize}
%
Next, we formalize our search for witnesses to type-errors.
%
We present the syntax and operational semantics of \lang -- a simple
lambda calculus with integers and booleans, extended with our notion of
holes -- as well as our search algorithm.
%
We prove that our system \emph{soundly} finds witnesses, \ie if we find
a witness then there is no possible typing for the input program.
%
\subsection{Syntax}
\label{sec:syntax}
\begin{figure}
% \hrule width 0.48\textwidth \vspace{0.05in}
$$
\begin{array}{rrcl}
% \emphbf{Configurations} \quad
%   & c & ::=    & \triple{e}{\vsu}{\tsu} \spmid \triple{\stuck}{\vsu}{\tsu} \\[0.05in]

\emphbf{Expressions}
  & \estuck & ::= & e \spmid \stuck \\
  & e & ::=    & v \spmid x \spmid \eapp{e}{e} \spmid \eplus{e}{e}\\
  &   & \spmid & \eif{e}{e}{e} \\
  % &   & \spmid & \elet{x}{e}{e} \\
  &   & \spmid & \epair{e}{e} \spmid \epcase{e}{x}{x}{e} \\
  &   & \spmid & \enode{e}{e}{e} \spmid \eleaf \\
  &   & \spmid & \ecase{e}{e}{x}{x}{x}{e} \\[0.1in]

\emphbf{Values}
  & v  & ::= & n \spmid b \spmid \efun{x}{e} \spmid \vhole{\thole} \spmid tr \\
  & tr & ::= & \vnode{t}{v}{v}{v} \spmid \vleaf{t} \\[0.05in]

\emphbf{Integers}
  & n & ::= &  0,1,-1,\ldots \\[0.05in]

\emphbf{Booleans}
  & b & ::= &  \etrue \spmid \efalse \\[0.05in]

\emphbf{Types}
  & t & ::=     & \tbool \spmid \tint \spmid \tfun \\
  &   &  \spmid & \tprod{t}{t} \spmid \ttree{t} \spmid \thole \\[0.05in]

\emphbf{Substitutions}
  & \vsu & ::= & \emptysu \spmid \extendsu{\vsu}{\vhole{\thole}}{v} \\
  & \tsu & ::= & \emptysu \spmid \extendsu{\tsu}{\thole}{t} \\[0.1in]
% \end{array}
% $$
% % \hrule width 0.48\textwidth
% $$
% \begin{array}{rrcl}
\emphbf{Contexts}
  & C
  & ::=
  &   	 \bullet
  \spmid \eapp{C}{e}
  \spmid \eapp{v}{C} \\
  & & \spmid & \eplus{C}{e} \spmid \eplus{v}{C} \\
  & & \spmid & \eif{C}{e}{e} \\
  % & & \spmid & \elet{x}{C}{e} \\
  & & \spmid & \epair{C}{e} \spmid \epair{v}{C} \\
  & & \spmid & \epcase{C}{x}{x}{e} \\
  & & \spmid & \enode{C}{e}{e} \\
  & & \spmid & \enode{v}{C}{e} \\
  & & \spmid & \enode{v}{v}{C} \\
  & & \spmid & \ecase{C}{e}{x}{x}{x}{e} \\[0.05in]

\emphbf{Type Contexts}
  & T &::=& \bullet \spmid \ttree{T} \spmid \tprod{T}{t} \spmid \tprod{t}{T} \\[0.05in]
\end{array}
$$

% \judgementHead{Reduction}{\eval{e}{e}}

% $$
% \begin{array}{rcl}
% \eval{C[e]&}{&C[e']} \qquad \text{if}\ \eval{e}{e'} \\
% 	\eval{\eapp{c}{v}&}{& \ceval{c}{v}}\\
% \eval{\eapp{(\efun{x}{\tau_x}{e})}{e_x}&}{&e\sub{x}{e_x}}\\
% 	\eval{\elet{x}{e_x}{e}&}{&e\sub{x}{e_x}} \\
% 	\eval{\ecase{D_j\ \overline{e}}{D_i}{\overline{y_i}}{e_i}{x}&}
% 	{&e_j\sub{x}{D_j\ \overline{e}}\sub{\overline{y_j}}{\overline{e}}} \\
% \end{array}
% $$

\caption{Syntax of \lang}
\label{fig:syntax}
\end{figure}

%
Figure~\ref{fig:syntax} describes the syntax of \lang, a simple lambda
calculus with numbers and booleans.
%
As we are specifically interested in
programs that \emph{do} go wrong, we include an explicit \stuck state in
our syntax.
%
\paragraph{Holes}
\label{sec:holes}
The main novelty in our system is the notion of a ``hole'', written
\ehole{}, which represents an uninstantiated value.
%
Importantly, we do not even know what type the value should have.
%
Holes may also appear in types, where they may be thought of as type
variables that we will not generalize over.
%
\subsection{Semantics}
\label{sec:semantics}
\begin{figure*}
\judgementHead{Evaluation}{\step{e}{\su}{e}{\su}}
\begin{gather*}
\inference[\recontext]
  {\step{e}{\su}{e_1}{\su_1}}
  {\step{C[e]}{\su}{C[e_1]}{\su_1}}
\qquad
\inference[\restuck]
  {}
  {\step{C[\stuck]}{\su}{\stuck}{\su}}
\\ \\
\inference[\replusgood]
  {\pair{n_1}{\su_2} = \force{v_1}{\tint} \\
   \pair{n_2}{\su_3} = \force{v_2}{\tint} \\ 
   n = \eplus{n_1}{n_2}}
  {\step{\eplus{v_1}{v_2}}{\su_1}{n}{\su_1;\su_2;\su_3}}
\qquad
\inference[\replusbadone]
  {\pair{\stuck}{\su_2} = \force{v_1}{\tint}}
  {\step{\eplus{v_1}{v_2}}{\su_1}{\stuck}{\su_1;\su_2}}
\\ \\
\inference[\replusbadtwo]
  {\pair{\stuck}{\su_2} = \force{v_2}{\tint}}
  {\step{\eplus{v_1}{v_2}}{\su_1}{\stuck}{\su_1;\su_2}}
\qquad
\inference[\reifgoodone]
  {\pair{\etrue}{\su_2} = \force{v}{\tbool}}
  {\step{\eif{v}{e_1}{e_2}}{\su_1}{e_1}{\su_1;\su_2}}
\\ \\
\inference[\reifgoodtwo]
  {\pair{\efalse}{\su_2} = \force{v}{\tbool}}
  {\step{\eif{v}{e_1}{e_2}}{\su_1}{e_2}{\su_1;\su_2}}
\qquad
\inference[\reifbad]
  {\pair{\stuck}{\su_2} = \force{v}{\tbool}}
  {\step{\eif{v}{e_1}{e_2}}{\su_1}{\stuck}{\su_1;\su_2}}
\\ \\
\inference[\reappgood]
  {\pair{\efun{x}{e}}{\su_2} = \force{v_1}{\tfun{\thole{}}{\thole{}}}}
  {\step{\eapp{v_1}{v_2}}{\su_1}{e\sub{x}{v_2}}{\su_1;\su_2}}
\qquad
\inference[\reappbad]
  {\pair{\stuck}{\su_2} = \force{v_1}{\tfun{\thole{}}{\thole{}}}}
  {\step{\eapp{v_1}{v_2}}{\su_1}{\stuck}{\su_1;\su_2}}
\\ \\
\inference[\relet]
  {}
  {\step{\elet{x}{v}{e}}{\su}{e\sub{x}{v}}{\su}}
\end{gather*}
\\ % [0.05in]
\relDescription{\forcesym and \gensym}
\begin{gather*}
\begin{array}{lcl}
\force{\ehole{i}}{t} & \defeq & \elet{v}{\gen{t}}{\pair{v}{\ehole{i} \mapsto v}} \\
\force{v}{\ehole{}}  & \defeq & \pair{v}{\emptysu} \\
\force{n}{\tint}    & \defeq & \pair{n}{\emptysu} \\
\force{v}{\tint}    & \defeq & \pair{\stuck}{\emptysu} \\
\force{b}{\tbool}   & \defeq & \pair{b}{\emptysu} \\
\force{v}{\tbool}   & \defeq & \pair{\stuck}{\emptysu} \\
\force{\efun{x}{e}}{\tfun{\thole{}}{\thole{}}} & \defeq & \pair{\efun{x}{e}}{\emptysu} \\
\force{v}{\tfun{\thole{}}{\thole{}}} & \defeq & \pair{\stuck}{\emptysu} \\
\end{array}
\qquad
\begin{array}{lcll}
\gen{\tint}   & \defeq & n & \\
\gen{\tbool}  & \defeq & b & \\
\gen{\tfun{t_1}{t_2}} & \defeq & \efun{x}{\ehole{i}}, & \quad \text{$i$ is fresh} \\
\gen{\thole{}} & \defeq & \ehole{i}, & \quad \text{$i$ is fresh} \\
\end{array}
\end{gather*}
\caption{Evaluation relation}
\label{fig:operational}
\end{figure*}

%
Figure~\ref{fig:operational} describes the small-step contextual
reduction semantics for \lang.
%
We write \stepi{j}{e}{\su}{e'}{\su'} if there exist $e_1,\ldots,e_j$ such that
$e$ is $e_1$, $e'$ is $e_j$ and $\forall i,j, 1 \leq i < j$, we have
\step{e_i}{\su_i}{e_{i+1}}{\su_{i+1}}.
%
We write \steps{e}{\su}{e'}{\su'} if there exists some (finite) $j$ such that
$\stepi{j}{e}{\su}{e'}{\su'}$.
%
The evaluation relation is parameterized by a pair of \forcesym and
\gensym functions that handle runtime type-checking and instantiation of
holes respectively.
%
The relation must also maintain a substitution \su, mapping holes to
concrete values, so that we do not instantiate the same hole with
different values in different contexts, and so that we can report a
concrete witness to any discovered type errors.

Note that each primitive reduction step -- addition, if-elimination, and
function application -- uses \forcesym to ensure that values have the
appropriate type (and that holes are instantiated) before continuing the
computation.
%
Additionally, beta-reduction \emph{does not} type-check its argument, it
only ensures that the value being applied is a function.
%
\begin{thm}
\label{thm:all-reduce}
  Every closed expression $e$ reduces to a value $v$ (which may be \stuck).
  \ES{do we really need to state this, or is it obvious?}
\end{thm}
% \begin{proof}%[Proof of \autoref{thm:all-reduce}]
%   Simple induction on the evaluation relation.
% \end{proof}
%
\subsection{Soundness}
\label{sec:soundness}
We now show that our evaluation relation conservatively instantiates
holes, \ie if, given a function $f$, we report that
$\steps{\eapp{f}{\ehole{}}}{\emptysu}{\stuck}{\su}$,
then for any type $\tfun{s}{t}$ you assign to $f$, there exists an input \hastype{v}{s} such that
$\steps{\eapp{f}{v}}{\emptysu}{\stuck}{\su}$.

We can think of the evaluation of \eapp{f}{\ehole{}} as computing a
\emph{partial type} -- a type that may contain holes -- for $f$.
%
We can extract this type from the result of evaluation as follows.
%
\begin{defn}
\label{defn:partial-type}
  If \stepi{i}{\eapp{f}{\ehole{}}}{\emptysu}{e}{\su}, then the $i$th
  \emph{partial type} of f, written \ptype{i}{f},
  is \tfun{\typeof{\subst{\su}{\ehole{}}}}{\typeof{e}}.

  We will omit the subscript when we wish to refer to the final partial
  type.
\end{defn}
%
The \typeof{} function is an approximation of the type of an expression.
\begin{defn}
\label{def:typeof}
  \[
  \begin{array}{lcll}
    \typeof{n}   & \defeq & \tint & \\
    \typeof{b}   & \defeq & \tbool & \\
    \typeof{\efun{x}{e}} & \defeq & \tfun{\thole{i}}{\typeof{e}}, & \quad \text{$i$ is fresh} \\
    \typeof{e} & \defeq & \thole{i}, & \quad \text{$i$ is fresh} \\
  \end{array}
  \]
\end{defn}
%
We also define a compatibility relation between types.
%
\begin{defn}
\label{defn:type-compat}
  A type $s$ is \emph{compatible} with a type $t$, written \tcompat{s}{t}, if
  $\exists \su.\ t = \subst{\su}{s} \lor s = \subst{\su}{t}$.
  \ES{we're abusing the \su notation here..}
\end{defn}
%
Given these two definitions, we show that each evaluation step
refines the partial type of $f$.
%
\begin{lem}
\label{lem:refine-partial}
  For all $k$, \tcompat{\ptype{k}{f}}{\ptype{k+1}{f}}.
\end{lem}
\begin{proof}
  By case analysis on the evaluation rules.
  %
  Note that all rules preserve partial types with the exception of when
  \forcesym is called on a hole, which case we instantiate the hole with
  a concrete value.
  %
  But \hastype{\ehole{}}{\thole{}}, which is compatible with any type.
\end{proof}
%
Furthermore, only a call to \forcesym can change the partial type of $f$.
%
\begin{lem}
\label{lem:force-inst}
  For all $k$, if $\ptype{k}{f} \neq \ptype{k+1}{f}$, then \forcesym must
  have been called at step $k+1$.
\end{lem}
\begin{proof}
  By case analysis on the evaluation rules.
  %
  If $\ptype{k}{f} \neq \ptype{k+1}{f}$ then one of the holes in $f$'s
  argument must have been instantiated with a concrete value at step
  $k+1$.
  %
  An examination of the rules shows that only place this happens is
  in the first case of \forcesym.
\end{proof}
%
Finally, we show that any value that is not compatible with the $k$th
partial type of $f$ will cause $f$ to get stuck in \emph{at most} $k$
steps.
%
\begin{lem}
\label{lem:k-stuck}
  For any \hastype{v}{t} that is not compatible with
  the input of \ptype{k}{f}, \stepi{k}{\eapp{f}{v}}{\emptysu}{\stuck}{\su}.
\end{lem}
\begin{proof}
  By induction on $k$.

  Let $\tfun{s_k}{\_} = \ptype{k}{f}$. Suppose \hastype{v}{\tincompat{t}{s_{k+1}}}, we
  will show that \stepi{k+1}{\eapp{f}{v}}{\emptysu}{\stuck}{\su}.
  \begin{description}
  \item[Case \tincompat{t}{s_k}:]
    The inductive hypothesis applies.
  \item[Case \tcompat{t}{s_k} but \tincompat{t}{s_{k+1}}:]
    By Lemma~\ref{lem:force-inst} we must have called \forcesym at step
    $k+1$.
    %
    A case analysis of the applicable rules shows that \forcesym cannot
    have succeeded.
  \end{description}
\end{proof}
%
Now we can prove our soundness theorem.
%
\begin{thm}
\label{thm:soundness}
  For any function $f$, if \steps{\eapp{f}{\ehole{}}}{\emptysu}{\stuck}{\su},
  then $\forall t. \exists \hastype{v}{t}. \steps{\eapp{f}{v}}{\emptysu}{\stuck}{\su}$.
\end{thm}
\begin{proof}
  Let $\tfun{s}{\_} = \ptype{}{f}$.
  \begin{description}
  \item[Case \tcompat{s}{t}:] Our witness is already valid.
  \item[Case \tincompat{s}{t}:] We can apply Lemma~\ref{lem:k-stuck} and
    use \emph{any} \hastype{v}{t} as a witness for $f$.
  \end{description}
\end{proof}

\subsection{Search Algorithm}
\label{sec:search-algorithm}
%
We have shown how to find a witness for a function of a single argument,
but in a language with higher-order functions and currying it may not be
clear \emph{syntactically} how many arguments a function takes.
%
Thus, we wrap our operational semantics for \lang in a search loop that
supplies an increasing number of arguments until the function returns a
value that is not a lambda.

The top-level search loop takes as input an open program -- a sequence
of binders -- and searches for an expression that closes the input
program and gets stuck. Concretely, given an input
%
\begin{code}
  let f1 = e1 in
  let f2 = e2 in
  ...
  let fn = en in
\end{code}
%
we will search for an expression of the form:
%
\begin{code}
  fn v1 ... vn
\end{code}
%
Figure~\ref{fig:expression-api} describes a small API for manipulating
and evaluating \lang expressions, which we will use to define our search
algorithm.
%
The bulk of the search is performed by @eval@, which
implements our operational semantics from \S\ref{sec:semantics}.
%
The operational semantics is non-deterministic due to \gensym,
thus @eval@ returns a list of possible results.
%
\begin{figure}[t]
  \centering
  \begin{mcode}
  -- transitive small-step reduction,
  -- returning a list of results
  eval :: ($e$, $\su$) -> [($v$, $\su$)]

  -- manipulating expressions
  subst   :: $\su$ -> $v$ -> $v$
  mkApps  :: $e$ -> [$e$] -> $e$
  mkLets  :: [($x$, $e$)] -> $e$ -> $e$
  isStuck :: $v$ -> Bool
  \end{mcode}
  \caption{Expression API}
  \label{fig:expression-api}
\end{figure}
%
We also define a few helper functions for manipulating expressions:
\begin{itemize}
\item @subst@ applies a substitution of holes to a value,
\item @mkApps@ creates a nested sequence of applications in the usual
  left-associative style,
\item @mkLets@ takes a list of binders and a body expression, and
  creates a sequence of nested let-binders, and
\item @isStuck@ tests whether a value is the \stuck term.
\end{itemize}
%
\begin{figure*}[t]
  \centering
  \begin{mcode}
  check :: [($x$, $e$)] -> Result
  check bnds =
    -- (2) search for a witness
    case find (isStuck . fst) results of
      Nothing      -> Safe
      Just (_, su) -> Unsafe (mkApps f (subst su args))
    where
      (args, results) = loop []
      f               = snd (last bnds)
      build args      = mkLets bnds (mkApps f args)

      -- (1) find the correct number of arguments
      loop :: [$v$] -> ([$v$], [($v$, $\su$)])
      loop args = case eval (build args, []) of
        ($\efun{x}{e}$, _) : _ -> loop (args `snoc` $\ehole{}$)
        results      -> (args, results)
  \end{mcode}
  \caption{Implementing our counter-example search in terms of the
    operational semantics.
    \ES{1 and 2 are backwards, ugh...}
    \ES{should address case where output types of successive runs dont match}
  }
  \label{fig:search-algo}
\end{figure*}
%
Figure~\ref{fig:search-algo} summarizes the overall implementation of
our search for witnesses, which takes as input a sequences of binders
and returns either @Safe@ if no witness could be found, or |Unsafe $e$|
where $e$ is a term that causes the input program to get stuck.
%
The search is split into two phases: (1) we supply an increasing number
of arguments until we find a saturated application, and (2) we search
through the list of results from the non-deterministic evaluator for a
witness.


% Algorithm:
%   Input: ML Program (let f1 = e1 in let f2 = e2 in ... e)
%   Output: 'safe' or 'fn v1 .. vn' where 'fn v1 .. vn' gets stuck
