
\begin{itemize}
\item how do we run ill-typed programs?
\item for a lang like ocaml, dynamic semantics are independent of static
  semantics, just lambda calculus. so no problem to run ill-typed
  program
\item but what about functions? what type of arguments should we pass? consider

\begin{lstlisting}
let f x = 
  let y = 1 + x in
    1. +. y
\end{lstlisting}

does \texttt{f} take an int, float, string? int and float are both
somewhat plausible, but string or anything else is ``clearly'' bogus. so
we cannot provide \emph{completely arbitrary} inputs to
\texttt{f}. Instead, we call \texttt{f} with a \emph{hole}, written
\ehole{}, which is a placeholder for a value whose type we have not
yet determined. As we execute the program, we instantiate holes with
concrete values as demanded by the primitive operations in the
program. For example, the hole we pass to f will be instantiated to an
int when we reach the \lstinline{1 + x} term. Thus, y will be an int as
well, and the program will get stuck at \lstinline{1. +. y}. \ES{this
  reads more like overview text..}

Next, we formalize our search for witnesses to type-errors. We present
the syntax and dynamic semantics of a simple lambda calculus with
integers and booleans, extended with our notion of holes, as well as our
search algorithm. We prove that our system \emph{soundly} finds
witnesses, \ie if we find a witness then there is no possible typing for
the input program.

\subsection{Syntax}
\label{sec:syntax}
\begin{figure}
% \hrule width 0.48\textwidth \vspace{0.05in}
$$
\begin{array}{rrcl}
% \emphbf{Configurations} \quad
%   & c & ::=    & \triple{e}{\vsu}{\tsu} \spmid \triple{\stuck}{\vsu}{\tsu} \\[0.05in]

\emphbf{Expressions}
  & \estuck & ::= & e \spmid \stuck \\
  & e & ::=    & v \spmid x \spmid \eapp{e}{e} \spmid \eplus{e}{e}\\
  &   & \spmid & \eif{e}{e}{e} \\
  % &   & \spmid & \elet{x}{e}{e} \\
  &   & \spmid & \epair{e}{e} \spmid \epcase{e}{x}{x}{e} \\
  &   & \spmid & \enode{e}{e}{e} \spmid \eleaf \\
  &   & \spmid & \ecase{e}{e}{x}{x}{x}{e} \\[0.1in]

\emphbf{Values}
  & v  & ::= & n \spmid b \spmid \efun{x}{e} \spmid \vhole{\thole} \spmid tr \\
  & tr & ::= & \vnode{t}{v}{v}{v} \spmid \vleaf{t} \\[0.05in]

\emphbf{Integers}
  & n & ::= &  0,1,-1,\ldots \\[0.05in]

\emphbf{Booleans}
  & b & ::= &  \etrue \spmid \efalse \\[0.05in]

\emphbf{Types}
  & t & ::=     & \tbool \spmid \tint \spmid \tfun \\
  &   &  \spmid & \tprod{t}{t} \spmid \ttree{t} \spmid \thole \\[0.05in]

\emphbf{Substitutions}
  & \vsu & ::= & \emptysu \spmid \extendsu{\vsu}{\vhole{\thole}}{v} \\
  & \tsu & ::= & \emptysu \spmid \extendsu{\tsu}{\thole}{t} \\[0.1in]
% \end{array}
% $$
% % \hrule width 0.48\textwidth
% $$
% \begin{array}{rrcl}
\emphbf{Contexts}
  & C
  & ::=
  &   	 \bullet
  \spmid \eapp{C}{e}
  \spmid \eapp{v}{C} \\
  & & \spmid & \eplus{C}{e} \spmid \eplus{v}{C} \\
  & & \spmid & \eif{C}{e}{e} \\
  % & & \spmid & \elet{x}{C}{e} \\
  & & \spmid & \epair{C}{e} \spmid \epair{v}{C} \\
  & & \spmid & \epcase{C}{x}{x}{e} \\
  & & \spmid & \enode{C}{e}{e} \\
  & & \spmid & \enode{v}{C}{e} \\
  & & \spmid & \enode{v}{v}{C} \\
  & & \spmid & \ecase{C}{e}{x}{x}{x}{e} \\[0.05in]

\emphbf{Type Contexts}
  & T &::=& \bullet \spmid \ttree{T} \spmid \tprod{T}{t} \spmid \tprod{t}{T} \\[0.05in]
\end{array}
$$

% \judgementHead{Reduction}{\eval{e}{e}}

% $$
% \begin{array}{rcl}
% \eval{C[e]&}{&C[e']} \qquad \text{if}\ \eval{e}{e'} \\
% 	\eval{\eapp{c}{v}&}{& \ceval{c}{v}}\\
% \eval{\eapp{(\efun{x}{\tau_x}{e})}{e_x}&}{&e\sub{x}{e_x}}\\
% 	\eval{\elet{x}{e_x}{e}&}{&e\sub{x}{e_x}} \\
% 	\eval{\ecase{D_j\ \overline{e}}{D_i}{\overline{y_i}}{e_i}{x}&}
% 	{&e_j\sub{x}{D_j\ \overline{e}}\sub{\overline{y_j}}{\overline{e}}} \\
% \end{array}
% $$

\caption{Syntax of \lang}
\label{fig:syntax}
\end{figure}


\subsection{Semantics}
\label{sec:semantics}
\begin{figure*}
\judgementHead{Evaluation}{\step{e}{\su}{e}{\su}}
\begin{gather*}
\inference[\recontext]
  {\step{e}{\su}{e_1}{\su_1}}
  {\step{C[e]}{\su}{C[e_1]}{\su_1}}
\qquad
\inference[\restuck]
  {}
  {\step{C[\stuck]}{\su}{\stuck}{\su}}
\\ \\
\inference[\replusgood]
  {\pair{n_1}{\su_2} = \force{v_1}{\tint} \\
   \pair{n_2}{\su_3} = \force{v_2}{\tint} \\ 
   n = \eplus{n_1}{n_2}}
  {\step{\eplus{v_1}{v_2}}{\su_1}{n}{\su_1;\su_2;\su_3}}
\qquad
\inference[\replusbadone]
  {\pair{\stuck}{\su_2} = \force{v_1}{\tint}}
  {\step{\eplus{v_1}{v_2}}{\su_1}{\stuck}{\su_1;\su_2}}
\\ \\
\inference[\replusbadtwo]
  {\pair{\stuck}{\su_2} = \force{v_2}{\tint}}
  {\step{\eplus{v_1}{v_2}}{\su_1}{\stuck}{\su_1;\su_2}}
\qquad
\inference[\reifgoodone]
  {\pair{\etrue}{\su_2} = \force{v}{\tbool}}
  {\step{\eif{v}{e_1}{e_2}}{\su_1}{e_1}{\su_1;\su_2}}
\\ \\
\inference[\reifgoodtwo]
  {\pair{\efalse}{\su_2} = \force{v}{\tbool}}
  {\step{\eif{v}{e_1}{e_2}}{\su_1}{e_2}{\su_1;\su_2}}
\qquad
\inference[\reifbad]
  {\pair{\stuck}{\su_2} = \force{v}{\tbool}}
  {\step{\eif{v}{e_1}{e_2}}{\su_1}{\stuck}{\su_1;\su_2}}
\\ \\
\inference[\reappgood]
  {\pair{\efun{x}{e}}{\su_2} = \force{v_1}{\tfun{\thole{}}{\thole{}}}}
  {\step{\eapp{v_1}{v_2}}{\su_1}{e\sub{x}{v_2}}{\su_1;\su_2}}
\qquad
\inference[\reappbad]
  {\pair{\stuck}{\su_2} = \force{v_1}{\tfun{\thole{}}{\thole{}}}}
  {\step{\eapp{v_1}{v_2}}{\su_1}{\stuck}{\su_1;\su_2}}
\\ \\
\inference[\relet]
  {}
  {\step{\elet{x}{v}{e}}{\su}{e\sub{x}{v}}{\su}}
\end{gather*}
\\ % [0.05in]
\relDescription{\forcesym and \gensym}
\begin{gather*}
\begin{array}{lcl}
\force{\ehole{i}}{t} & \defeq & \elet{v}{\gen{t}}{\pair{v}{\ehole{i} \mapsto v}} \\
\force{v}{\ehole{}}  & \defeq & \pair{v}{\emptysu} \\
\force{n}{\tint}    & \defeq & \pair{n}{\emptysu} \\
\force{v}{\tint}    & \defeq & \pair{\stuck}{\emptysu} \\
\force{b}{\tbool}   & \defeq & \pair{b}{\emptysu} \\
\force{v}{\tbool}   & \defeq & \pair{\stuck}{\emptysu} \\
\force{\efun{x}{e}}{\tfun{\thole{}}{\thole{}}} & \defeq & \pair{\efun{x}{e}}{\emptysu} \\
\force{v}{\tfun{\thole{}}{\thole{}}} & \defeq & \pair{\stuck}{\emptysu} \\
\end{array}
\qquad
\begin{array}{lcll}
\gen{\tint}   & \defeq & n & \\
\gen{\tbool}  & \defeq & b & \\
\gen{\tfun{t_1}{t_2}} & \defeq & \efun{x}{\ehole{i}}, & \quad \text{$i$ is fresh} \\
\gen{\thole{}} & \defeq & \ehole{i}, & \quad \text{$i$ is fresh} \\
\end{array}
\end{gather*}
\caption{Evaluation relation}
\label{fig:operational}
\end{figure*}



\item values are tagged with their types, just like ``untyped'' langs
\item special ``hole'' value whose type is not yet known, used for function args
\item on-the-fly unification to determine ``correct'' type for holes
\end{itemize}

% Algorithm:
%   Input: ML Program (let f1 = e1 in let f2 = e2 in ... e)
%   Output: 'safe' or 'fn v1 .. vn' where 'fn v1 .. vn' gets stuck

