%
% LaTeX template for prepartion of submissions to PLDI'15
%
% Requires temporary version of sigplanconf style file provided on
% PLDI'15 web site.
% 
\documentclass[pldi,indentedstyle=false]{sigplanconf-pldi15}

%
% the following standard packages may be helpful, but are not required
%
\usepackage{SIunits}            % typset units correctly
\usepackage{courier}            % standard fixed width font
\usepackage[scaled]{helvet} % see www.ctan.org/get/macros/latex/required/psnfss/psnfss2e.pdf
\usepackage{url}                  % format URLs
\usepackage{listings}          % format code
\usepackage{enumitem}      % adjust spacing in enums
\usepackage[colorlinks=true,allcolors=blue,breaklinks,draft=false]{hyperref}   % hyperlinks, including DOIs and URLs in bibliography
% known bug: http://tex.stackexchange.com/questions/1522/pdfendlink-ended-up-in-different-nesting-level-than-pdfstartlink
\newcommand{\doi}[1]{doi:~\href{http://dx.doi.org/#1}{\Hurl{#1}}}   % print a hyperlinked DOI



\begin{document}

%
% any author declaration will be ignored  when using 'plid' option (for double blind review)
%

\title{NanoMaLy}

\authorinfo{Eric L. Seidel\and Ranjit Jhala}
           {UC San Diego}
           {{eseidel,rjhala}@cs.ucsd.edu}

\authorinfo{Westley Weimer}
           {University of Virginia}
           {weimer@viginia.edu}

\maketitle
\begin{abstract}
\end{abstract}


% 11 pages total (not including bib)
% ----------------------------------
%  1p : intro
%  2p : overview
%  2p : type-carrying semantics
%  1p : evaluation: quickchecking type errors
%  2p : interactive semantics
%  1p : evaluation: nanomaly
%  1p : related work
%  1p : conclusion

\section{Introduction}          % 1 page
\section{Overview}              % 2 pages?
\section{Type-carrying Semantics} % 2 pages
\begin{itemize}
\item values are tagged with their types, just like ``untyped'' langs
\item special ``hole'' value whose type is not yet known, used for function args
\item on-the-fly unification to determine ``correct'' type for holes
\end{itemize}

% Algorithm:
%   Input: ML Program (let f1 = e1 in let f2 = e2 in ... e)
%   Output: 'safe' or 'fn v1 .. vn' where 'fn v1 .. vn' gets stuck




\section{Evalution: Recasting Type Errors as Runtime Errors} % 1 page
\label{sec:eval-recast-type}
\begin{itemize}
\item benchmarks: our data + seminal data
\item both cases: \textbf{random} search sufficient to trigger runtime crash in 80\% of programs
\item how many of the ``safe'' programs are actually safe??
\end{itemize}

\section{``Interactive'' Semantics \& Trace Exploration} % 2 pages
\begin{itemize}
\item extend operational semantics to collect reduction graph
\item nodes are terms, edges indicate ``steps-to'' and ``sub-term'' relations
\item visualize path through reduction graph
\item expand edges to reveal more fine-grained steps (step/jump forward/backward)
\item never lose context (unlike traditional debugger)
\end{itemize}

\section{Evaluation: NanoMaLy}                % 1 page
\section{Related Work}              % 1 pages

\subsection{Type Error Localization}
\label{sec:type-error-local}
\begin{itemize}
\item seminal
\item bayesian (Zhang+Myers)
\item MaxSMT
\item counterfactual (Chen+Erwig)
\end{itemize}
\subsection{Debugging \& Program Understanding}
\label{sec:debugging}



\section{Conclusion}                % 1 page


\end{document}
