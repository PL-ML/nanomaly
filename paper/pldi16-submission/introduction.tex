\section{Introduction}
\label{sec:introduction}

% \paragraph{Problem}
Type errors are a common stumbling block for students trying to learn
typed functional languages like \ocaml and \haskell.
%
The compiler imposes a strict discipline that prevents novices from
running their programs until the type-checker has been appeased.
%
In effect this prioritizes the \emph{static} semantics of the language
over the \emph{dynamic} semantics.
%
This is fine for expert users who are already familiar with the language
and type system, but seems backwards for novices, who are typically
introduced to the dynamic semantics first.

Consider the following ill-typed @factorial@ function, where we return
@true@ in the base case instead of @1@.
%
\begin{code}
  let rec factorial n =
    if n <= 0 then
      true
    else
      n * factorial (n-1)
\end{code}
%
\ocaml's response is to tell us that on line 5, the recursive call to
@factorial@ returns a @bool@, but we needed an @int@ for the
multiplication.
%
\begin{verbatim}
File "fac.ml", line 5, characters 8-23:
Error: This expression has type bool
       but an expression was expected
       of type int
\end{verbatim}
%
\ocaml is correct, but we may ask \emph{why} @factorial@ returns a
@bool@ instead of an @int@.

Contrast this \emph{static} type error to the \emph{dynamic} type error
that arises in the equivalent \ruby function, when we call
@factorial(1)@.
%
\begin{verbatim}
fac.rb:5:in `*': true can't be coerced
                 into Fixnum (TypeError)
  from fac.rb:5:in `factorial'
  from fac.rb:10:in `<main>'
\end{verbatim}
%
The same basic information is there, @*@ was given a @bool@ (@true@)
when it expected an @int@ (@FixNum@), but since the error occurred at
runtime, \ruby is able to provide a stack trace that provides important
contextual information about the error.
%
More importantly, being able to \emph{run} the bad program means we can
also test it on other inputs, \eg @0@, or we can load it into a debugger
and step through the execution to see concretely how the error arises.

To make matters worse, it is well known that type inference algorithms
based on unification have a penchant for \emph{reporting} errors far
from their source.
%
This further increases the novice's confusion and can actively mislead
them to focus their investigation on an irrelevant piece of code.
%
Much work has been done on pinpointing the \emph{source} of a type
error~\cite{zhang-myers, maxsmt, counterfactual}, but an accurate source
location still does not explain \emph{why} the program is wrong.

% \paragraph{Contributions}
In this paper we propose exposing \emph{static} type errors for the
\emph{runtime} crashes that they are.
%
Our first contribution is an algorithm for searching for
\emph{witnesses} to type errors, \ie input vectors that cause a program
to crash with a type error.
%
This is not as trivial as it sounds.
%
While one can indeed execute an \ocaml program without type-checking it
-- the dynamic semantics do not depend on the static semantics -- what
types of values should one pass to the program?
%
We do not want to generate \emph{spurious} witnesses, inputs that the
type-checker would have excluded, so we introduce a semantics for
executing \ocaml programs with ``holes'', values whose type we do not
yet know.
%
% A hole can be passed around through the program until it is demanded by
% a primitive operation, \eg addition, function application, branching,
% etc., at which point we know from the context what type it must have,
% and can instantiate it with a concrete value.
%
Our semantics \emph{conservatively} instantiates holes with concrete
values, if we can find a witness to a type error then there is no valid
typing for the program.

Given a witness to a type error, the prospective \ocaml student may
still be at a loss.
%
The standard \ocaml interpreter and debugging infrastructure expect
well-typed programs, so they cannot be used to investigate \emph{how}
the witness causes the program to crash.
%
Thus, our second contribution is an interactive visualization of the
execution of \ocaml programs (well-typed or not).
%
We present the execution as a \emph{reduction graph}, in which the nodes
are expressions and the edges indicate either a single-step reduction or
a sub-term relation.
%
This graph is easy to generate during the execution and admits many of
the standard debugging steps, \eg ``step forward'', ``step into'', and
``step over'', as well as backwards motion and jumping to the next (or
previous) function call.
%
The visualization initially shows the source term (\ie the witness) and
the stuck term (or the final value), and allows the novice to
iteratively expand the computation to show intermediate steps by
selecting an expression and choosing an strategy for picking the next
node to insert into the graph.
%
This process of iterative expansion of the trace ensures that the
context of intermediate steps is not lost; whereas traditional debuggers
force one to move between statements, in our visualization you never
lose sight of where you came from.

\paragraph{Outline}
We begin the rest of the paper with an overview of our search algorithm
and trace visualization (\S~\ref{sec:overview}).
%
Then, we formalize our algorithm and prove it sound
(\S~\ref{sec:searching-witness}), and evaluate it on a large set of real
student programs (\S~\ref{sec:eval-witness}).
%
Next, we extend the search algorithm to collect the reduction graph and
formalize a set of traversals (\S~\ref{sec:explaining}), and measure the
complexity of these traces on our benchmarks
(\S~\ref{sec:evaluation-nanomaly}).
%
Finally, we present the results of a preliminary user study on the full
system (\S~\ref{sec:user-study}).

%%% Local Variables:
%%% mode: latex
%%% TeX-master: "main"
%%% End:
